\chapter{Funciones complejas}

Una función de una variable compleja es una aplicación que toma un conjunto ($D$) de números complejos y los transforma en otro conjunto ($S$) de números complejos $w \in S$.

\begin{mdframed}[style=DefinitionFrame]
    \begin{defn}
        \label{defn:complexFunc1}
    \end{defn}
    \cusTi{Función compleja}
    \begin{equation*}
        f: D \subseteq \setC \longrightarrow S \subseteq \setC \tq f(z) = w
    \end{equation*}
\end{mdframed}

Las funciones complejas no se pueden graficar, porque tanto los elementos del conjunto de partida como los elementos del conjunto de llegada son puntos del plano complejo.
Es por esto que para tener ciertas nociones geométricas de la función, podemos hacer dos gráficos en 2D de dichos conjuntos de partida y llegada.
Es posible graficar el módulo de una función en 3D.
Se usan dos variables independientes para el conjunto de partida y una dependiente para el conjunto de llegada.

Una vez aplicada una función, en ocaciones suele ser útil factorizar todos los términos que tienen multiplicado a $\iu$ de los que no para obtener la expresión binomial de la imagen.
\begin{equation*}
    w = u + \iu v
\end{equation*}

Así, las imágenes se pueden expresar como binomios complejos o como pares ordenados, sabiendo que basta con sumar la primer componente ($u$) más la segunda componente ($v$) multiplicada por $\iu$ para obtener nuevamente la forma binomial.
\begin{equation*}
    u + \iu v \equiv (u,v)
\end{equation*}

Siguiendo este criterio, suele usarse el vector $(x,y)$ de variables independientes para representar la variable compleja $x + \iu y$ como notación equivalente.
\begin{equation*}
    x + \iu y \equiv (x,y)
\end{equation*}

Pudiendo definir $\Vec{F}: \setR^2 \longrightarrow \setR^2$ y $f: \setR^2 \longrightarrow \setC$, ambas equivalentes a la función compleja dada en la definición \ref{defn:complexFunc1}.

\begin{mdframed}[style=DefinitionFrame]
    \begin{defn}
    \end{defn}
    \cusTi{Función compleja}
    \begin{equation*}
        f(x,y) = u(x,y) + \iu \, v(x,y)
    \end{equation*}
\end{mdframed}

\begin{mdframed}[style=DefinitionFrame]
    \begin{defn}
    \end{defn}
    \cusTi{Campo vectorial asociado a una función compleja}
    \begin{equation*}
        \Vec{F}(x,y) = \begin{bmatrix} u(x,y) & v(x,y) \end{bmatrix}
    \end{equation*}
\end{mdframed}


\section{Límite y continuidad}

\begin{mdframed}[style=DefinitionFrame]
    \begin{defn}
    \end{defn}
    \cusTi{Límite de una función compleja}
    \begin{multline*}
        \lim_{z \to z_0} f(z) = w_0 \iff \forall \varepsilon > 0 \enspace \exists \enspace \delta > 0 \tq
        \\
        0 < \norm{z-z_0} < \delta \Rightarrow \norm{f(z)-w_0} < \varepsilon
    \end{multline*}
\end{mdframed}

El límite de una función de variable compleja es en realidad un límite doble pués, según la propiedad \ref{prop:distance}, los módulos involucrados representan distancias en el plano complejo.

\begin{mdframed}[style=PropertyFrame]
    \begin{prop}
        \label{prop:limit}
    \end{prop}
    \begin{equation*}
        \lim_{z \to z_0} f(z) = \lim_{\substack{x \to x_0\\y \to y_0}} f(x + \iu y)
    \end{equation*}
\end{mdframed}

\begin{mdframed}[style=DefinitionFrame]
    \begin{defn}
    \end{defn}
    \cusTi{Continuidad}
    \cusTe{Una función compleja es continua si y solo si:}
    \begin{equation*}
        f(z_0)=\lim_{z \to z_0}f(z)
    \end{equation*}
\end{mdframed}

\begin{mdframed}[style=PropertyFrame]
    \begin{prop}
    \end{prop}
    El campo vectorial $\Vec{F}$ asociado a una función compleja $f$ es continuo si y solo si dicha función también lo es:
    \begin{equation*}
        \textrm{$f$ es continua $\iff \Vec{F}$ es continuo}
    \end{equation*}
\end{mdframed}

\begin{mdframed}[style=DefinitionFrame]
    \begin{defn}
    \end{defn}
    \cusTi{Función acotada}
    \cusTe{Una función compleja es acotada en un entorno reducido de un punto $(z_0)$ si existe algún valor real $(m,M \in \setR)$ tal que el módulo de la función sea mayor o menor a dicho valor.}
    \begin{gather*}
        \norm{f(z)} \geq m \quad \lor \quad \norm{f(z)} \leq M
        \\[1ex]
        \forall \quad z: 0 < \norm{z-z_0} < R
    \end{gather*}
\end{mdframed}

\begin{mdframed}[style=PropertyFrame]
    \begin{prop}
    \end{prop}
    \cusTi{Infinitésimo por acotado}
    \begin{gather*}
        \textrm{Dada} \enspace f(z)=g(z) \cdot h(z)
        \\
        \textrm{Si} \enspace g(z) \to 0 \enspace \land \enspace h(z) \enspace \textrm{es acotada}
        \\
        \textrm{Entonces} \enspace \lim_{z \to z_0} f(z) = 0
    \end{gather*}
\end{mdframed}


\section{Argumento}

El argumento de un número complejo es el ángulo que este forma con el eje $x$.

\begin{mdframed}[style=DefinitionFrame]
    \begin{defn}
    \end{defn}
    \cusTi{Argumento de un número complejo}
    \begin{equation*}
        \arg(z) = \arctan \inParentheses{\frac{y}{x}}
    \end{equation*}
\end{mdframed}

\begin{mdframed}[style=PropertyFrame]
    \begin{prop}
    \end{prop}
    \begin{equation*}
        \arg(z) = - \iu \ln \left( \frac{z}{\norm{z}} \right)
    \end{equation*}
\end{mdframed}

\begin{mdframed}[style=PropertyFrame]
    \begin{prop}
    \end{prop}
    El argumento de un complejo es el opuesto del argumento de su inverso.
    \begin{equation*}
        \arg(z) = - \arg(z^{-1})
    \end{equation*}
\end{mdframed}

\begin{mdframed}[style=PropertyFrame]
    \begin{prop}
    \end{prop}
    El argumento del producto de dos números es la suma de los argumentos de esos números.
    \begin{equation*}
        \arg (z_1 \, z_2) = \arg(z_1) + \arg(z_2)
    \end{equation*}
\end{mdframed}

\begin{mdframed}[style=PropertyFrame]
    \begin{prop}
    \end{prop}
    El argumento de la división de dos números es la resta de los argumentos de esos números.
    \begin{equation*}
        \arg \left( \dfrac{z_1}{z_2} \right) = \arg(z_1) - \arg(z_2)
    \end{equation*}
\end{mdframed}

Para poder definir funciones complejas, es necesario que cada elemento del dominio tenga solo una imagen.

Al representar un número complejo en su forma trigonométrica o exponencial es evidente que un mismo punto puede ser expresado mediante distintos números complejos por tener diferentes ángulos:
\begin{equation*}
    f(z_0) = w_k = \rho \, e^{\iu \left( \theta + 2 \kth \pi \right)} \quad \textrm{con} \enspace \kth \in \setZ
\end{equation*}

Por esto, para definir una función se restringen los valores que puede tomar el argumento de la imagen a un intervalo de $2\pi$ radianes a partir de cierto ángulo $\alpha$ arbitrario:
\begin{equation*}
    f(z) = w_\alpha \tq \arg(w_\alpha) \in [\alpha;\alpha+2\pi)
\end{equation*}

\begin{mdframed}[style=DefinitionFrame]
    \begin{defn}
    \end{defn}
    \cusTi{Argumento principal}
    \cusTe{Se lo define como aquel que toma un valor entre $-\pi$ y $\pi$ y se lo denota con mayúscula para distinguirlo, de modo que:}
    \begin{equation*}
        -\pi \leq \operatorname{Arg}(z) \leq \pi
    \end{equation*}
\end{mdframed}


\section{Exponenciales y logarítmicas}

Plantear una ecuación exponencial tiene como soluciones el conjunto $\sub{S}{exp}$ de los $w \in \setC$ que resultan de elevar el número $e$ a una potencia compleja $z \in \setC$.
\begin{equation*}
    \sub{S}{exp} = \inBraces{w \in \setC\ \tq e^z = w} = e^z
\end{equation*}

\begin{mdframed}[style=PropertyFrame]
    \begin{prop}
    \end{prop}
    Sea $w = e^z$ dado que $e^{x + \iu y} = e^x \, e^{\iu y}$ se tiene:
    \begin{equation*}
        \norm{w} = e^x \quad \land \quad \arg(w) = y + 2 \kth \pi
    \end{equation*}
\end{mdframed}

De forma similar a la ecuación exponencial, para el logaritmo neperiano se puede plantear el conjunto de números $w \in \setC$ que, de estar en el exponente del número $e$, dan como resultado $z \in \setC$.
\begin{equation*}
    \sub{S}{ln} = \inBraces{w \in \setC \tq e^w = z} = \ln(z)
\end{equation*}

Expresando un número complejo en su forma exponencial y aplicando propiedades logarítmicas
\begin{align*}
    z &= \norm{z} \, e^{\iu \arg(z)}
    \\
    \ln (z) &= \ln \left( \norm{z} \, e^{\iu \arg(z)} \right)
\end{align*}

Se obtiene la siguiente expresión alternativa para el logaritmo.

\begin{mdframed}[style=PropertyFrame]
    \begin{prop}
    \end{prop}
    \begin{equation*}
        \ln(z) = \ln \norm{z} + \iu \left( \theta + 2 \kth\pi \right) \quad \textrm{con} \enspace k \in \setZ
    \end{equation*}
\end{mdframed}


\section{Trigonométricas}

Partiendo de la propiedad de Euler (Prop. \ref{prop:EulerFormula}) y por la propiedad \ref{prop:ReIm} se tiene:
\begin{gather*}
    \left\{
    \begin{aligned}
        w &= e^{\iu z} = \cos(z) + \iu \sin(z)
        \\[1ex]
        \conj{w} &= e^{-\iu z} = \cos(z) - \iu \sin(z)
    \end{aligned}
    \right.
    \\[1em]
    \left\{
    \begin{aligned}
        w + \conj{w} &= e^{\iu z} + e^{-\iu z} = 2 \cos(z)
        \\[1ex]
        w - \conj{w} &= e^{\iu z} - e^{-\iu z} = 2 \iu \sin(z)
    \end{aligned}
    \right.
\end{gather*}

Despejando las ecuaciones del sistema anterior quedan definidas expresiones para el seno y el coseno de números complejos.

\begin{mdframed}[style=PropertyFrame]
    \begin{prop}
    \end{prop}
    \cusTi{Coseno complejo}
    \begin{equation*}
        \cos(z) = \frac{e^{\iu z} + e^{-\iu z}}{2}
    \end{equation*}
\end{mdframed}

\begin{mdframed}[style=PropertyFrame]
    \begin{prop}
    \end{prop}
    \cusTi{Seno complejo}
    \begin{equation*}
        \sin(z) = \frac{e^{\iu z} - e^{-\iu z}}{2 \iu}
    \end{equation*}
\end{mdframed}

O bien, expresando $z$ en su forma binomial
\begin{equation*}
    \left\{
    \begin{aligned}
        \cos(z) = \frac{e^{\iu \left(x + \iu y \right)} + e^{-\iu \left(x + \iu y \right)}}{2}
        \\[1ex]
        \sin(z) = \frac{e^{\iu \left(x + \iu y \right)} - e^{-\iu \left(x+\iu y\right)}}{2 \iu}
    \end{aligned}
    \right.
\end{equation*}

Si $x \neq 0 \land y \neq 0 \Rightarrow z \in \setC$ quedan expresiones alternativas para el coseno y el seno de números complejos:
\begin{equation*}
    \left\{
    \begin{aligned}
        \cos(z) &= \frac{e^{\iu x - y} + e^{-\iu x + y}}{2}
        \\[1ex]
        \sin(z) &= \frac{e^{\iu x - y} - e^{-\iu x + y}}{2\iu}
    \end{aligned}
    \right.
\end{equation*}

Si $y=0 \Rightarrow z \in \setR$ quedan expresiones para el coseno y el seno de números reales:
\begin{equation*}
    \left\{
    \begin{aligned}
        \cos(x) &= \frac{e^{\iu x} + e^{-\iu x}}{2}
        \\[1ex]
        \sin(x) &= \frac{e^{\iu x} - e^{-\iu x}}{2\iu}
    \end{aligned}
    \right.
\end{equation*}

Y si $x=0 \Rightarrow z \in \setI$ quedan expresiones para el coseno hiperbólico y el seno hiperbólico de números imaginarios puros:
\begin{equation*}
    \left\{
    \begin{aligned}
        \cos(\iu y) &= \frac{e^{-y} + e^y}{2} = \cosh(\iu y)
        \\[1ex]
        \sin(\iu y) &= \frac{e^{-y} - e^y}{2\iu} = -\iu \sinh{(\iu y)}
    \end{aligned}
    \right.
\end{equation*}