\chapter{Derivadas de funciones complejas}

\begin{mdframed}[style=DefinitionFrame]
    \begin{defn}
    \end{defn}
    \cusTi{Derivada}
    \cusTe{La derivada de $f(z)$ en $z_0 \in \setC$ se define como:}
    \begin{align*}
        \frac{\dif}{\dif z} f(z_0) &= \lim_{z \to z_0} \frac{f(z) - f(z_0)}{z - z_0}
        \\[1ex]
        &= \lim_{h \to 0} \frac{f(z_0+h) - f(z_0)}{h}
    \end{align*}
\end{mdframed}

Nótese que tanto la variable $z$ como el punto $z_0$ son complejos.
Por lo tanto el incremento $h=z-z_0$ también lo es.

\begin{center}
    \def\svgwidth{0.6\linewidth}
    \input{./images/calc-deriv.pdf_tex}
\end{center}

\begin{mdframed}[style=PropertyFrame]
    \begin{prop}
        \label{prop:partialDerivate}
    \end{prop}
    La derivada toma el mismo valor en todas las direcciones $(\Vec{r})$ del plano y particularmente en los ejes real e imaginario.
    \begin{equation*}
        \frac{\dif}{\dif z} f(z_0) = \frac{\partial}{\partial x} f(z_0) = \frac{\partial}{\partial y} f(z_0) = \frac{\partial}{\partial \Vec{r}} f(z_0)
    \end{equation*}
\end{mdframed}

Por la propiedad \ref{prop:limit} del límite, la derivada es un límite doble:

\begin{mdframed}[style=PropertyFrame]
    \begin{prop}
    \end{prop}
    \begin{gather*}
        \frac{\dif}{\dif z} f(z_0) = \lim_{\substack{x \to x_0\\y \to y_0}} \frac{f(x + \iu y) - f(x_0 + \iu y_0)}{(x-x_0) + \iu (y-y_0)}
        \\[1em]
        = \lim_{\substack{h_x \to 0\\h_y \to 0}}\frac{f \Big( (x_0+h_x) + \iu (y_0+h_y) \Big) - f(x_0 + \iu y_0)}{h_x + \iu h_y}
    \end{gather*}
\end{mdframed}


\section{Funciones analíticas}

\begin{mdframed}[style=DefinitionFrame]
    \begin{defn}
    \end{defn}
    \cusTi{Función analítica}
    \cusTe{Una función se dice analítica en $z_0$ si es derivable en un entorno abierto del punto.}
\end{mdframed}

\begin{mdframed}[style=PropertyFrame]
    \begin{prop}
    \end{prop}
    Si una función es analítica, es de clase $f \in \class[\infty]$ y sus derivadas sucesivas son analíticas.
\end{mdframed}

\begin{mdframed}[style=PropertyFrame]
    \begin{defn}
    \end{defn}
    \cusTi{Función entera}
    \cusTe{Una función se dice entera si es analítica en todo el plano complejo.}
\end{mdframed}

\begin{mdframed}[style=PropertyFrame]
    \begin{prop}
    \end{prop}
    Si una función es entera y acotada, entonces es constante.
\end{mdframed}

\begin{mdframed}[style=PropertyFrame]
    \begin{defn}
    \end{defn}
    \cusTi{Función armónica}
    \cusTe{Una función se dice armónica si es un campo escalar real $u: \setR^2 \longrightarrow \setR$ que verifica $\grad^2 u = 0$.
    Esto es, si cumple:}
    \begin{equation*}
        \frac{\partial^2}{\partial x^2} u(x,y) + \frac{\partial^2}{\partial y^2} u(x,y) = 0
    \end{equation*}
\end{mdframed}

\begin{mdframed}[style=PropertyFrame]
    \begin{prop}
    \end{prop}
    Las componentes $u(x,y)$ y $v(x,y)$ del campo vectorial asociado a una función compleja son armónicas solo si la función es entera.
\end{mdframed}

\begin{mdframed}[style=PropertyFrame]
    \begin{prop}
    \end{prop}
    Si un campo de gradientes $\Vec{F}=\grad u(x,y)$ verifica $\operatorname{div}(\Vec{F})=0$ entonces la función potencial $u(x,y)$ es armónica.
    Esto es:
    \begin{equation*}
        \grad^2 u(x,y) = u_{xx}'' + u_{yy}'' = 0
    \end{equation*}
\end{mdframed}

\begin{mdframed}[style=PropertyFrame]
    \begin{prop}
    \end{prop}
    Las curvas de nivel de $v(x,y)$ son las líneas de flujo del campo del cual $u(x,y)$, en caso de ser armónica, es el potencial.
\end{mdframed}

\begin{mdframed}[style=PropertyFrame]
    \begin{prop}
    \end{prop}
    Si $u: D \subseteq \setC \longrightarrow \setC$ está definida en un conjunto $D$ simplemente conexo, entonces $u(x,y)$ es armónica, existe la función $v(x,y)$ conjugada armónica de $u(x,y)$ y, además, $f(x,y)=u(x,y)+iv(x,y)$ es entera.
\end{mdframed}

\concept{Demostración}

Por las ecuaciones de Cauchy-Riemann (Def. \ref{defn:CRequations}) y por el teorema de Cauchy-Schwarz, se tiene:
\begin{gather*}
    \left\{
    \begin{aligned}
        u_x' &= v_y'
        \\
        u_y' &= -v_x'
    \end{aligned}
    \right.
    \\[1ex]
    \left\{
    \begin{aligned}
        u_{xx}'' &= v_{yx}''
        \\
        u_{yy}'' &= -v_{xy}''
    \end{aligned}
    \right.
    \\[1ex]
    u_{xx}'' + u_{yy}''= v_{yx} - v_{xy} = 0
    \\
    \therefore u \enspace \textrm{es armónica}
    \\[1ex]
    \left\{
    \begin{aligned}
        u_x' &= v_y'
        \\
        u_y' &= -v_x'
    \end{aligned}
    \right.
    \\[1ex]
    \left\{
    \begin{aligned}
        u_{xy}'' &= v_{yy}''
        \\
        u_{yx}'' &= - v_{xx}''
    \end{aligned}
    \right.
    \\[1ex]
    u_{yx}'' - u_{xy}'' = - v_{xx} - v_{yy} = 0
    \\
    \therefore v \enspace \textrm{es la conjugada armónica de} \enspace u
\end{gather*}


\section{Teorema de Cauchy-Riemann}

El teorema de Cauchy-Riemann relaciona la derivada de una función con las derivadas parciales de los campos escalares $u(x,y)$ y $v(x,y)$ componentes del campo vectorial asociado a la función.
Si una función es derivable entonces se cumple la siguiente propiedad.

\begin{mdframed}[style=PropertyFrame]
    \begin{prop}
        \label{prop:CRtheorem}
    \end{prop}
    \begin{equation*}
        \frac{\dif}{\dif z} f(z_0)= \frac{\partial}{\partial x} f(x_0,y_0)= -\iu \, \frac{\partial}{\partial y} f(x_0,y_0)
    \end{equation*}
\end{mdframed}

Nótese que las derivadas de la ecuación anterior son las parciales de los campos escalares $u(x,y)$ y $v(x,y)$, componentes del campo vectorial asociado.
Es importante no confundirlas con la derivada (\ref{prop:partialDerivate}) en las direcciones de los ejes real e imaginario.

Aplicando la propiedad \ref{prop:CRtheorem} a una función compleja expresada en la forma binomial, podemos obtener dos ecuaciones de gran relevancia.
\begin{gather*}
    \left\{
    \begin{aligned}
        \frac{\partial}{\partial x} f(x,y) &= \frac{\partial}{\partial x} u(x,y) + \iu \frac{\partial}{\partial x} v(x,y)
        \\
        \frac{\partial}{\partial y} f(x,y) &= \frac{\partial}{\partial y} u(x,y) + \iu \frac{\partial}{\partial y} v(x,y)
    \end{aligned}
    \right.
    \\[1ex]
    \frac{\partial}{\partial x} u(x,y) + \iu \frac{\partial}{\partial x} v(x,y) = - \iu \left( \frac{\partial}{\partial y} u(x,y) + \iu \frac{\partial}{\partial y} v(x,y) \right)
    \\[1ex]
    \frac{\partial}{\partial x} u(x,y) + \iu \frac{\partial}{\partial x} v(x,y) = \frac{\partial}{\partial y} v(x,y) - \iu \frac{\partial}{\partial y} u(x,y)
\end{gather*}

\begin{mdframed}[style=DefinitionFrame]
    \begin{defn}
        \label{defn:CRequations}
    \end{defn}
    \cusTi{Ecuaciones de C-R}
    \cusTe{Si una función es derivable entonces se cumple que:}
    \begin{align*}
        \frac{\partial}{\partial x} u(x,y) &= \frac{\partial}{\partial y} v(x,y)
        \\[1em]
        \frac{\partial}{\partial x} v(x,y) &= - \frac{\partial}{\partial y} u(x,y)
    \end{align*}
\end{mdframed}

\begin{mdframed}[style=PropertyFrame]
    \begin{prop}
    \end{prop}
    Si el campo vectorial asociado a una función compleja es diferenciable y se verifican las ecuaciones de Cauchy-Riemann, entonces la función es derivable.
\end{mdframed}

Aplicando una transformación biyectiva $(T)$ al campo vectorial $(\Vec{F})$ asociado a una función compleja $(f)$, al calcular la matriz de derivadas, o matriz Jacobiana $(J)$, se pueden deducir las relaciones entre las derivadas parciales en coordenadas polares.

Dada la siguiente función compleja y su campo vectorial asociado:
\begin{equation*}
    f(z) = u(z) + \iu v(z) \iff \Vec{F} = \begin{bmatrix} u(x,y) & v(x,y) \end{bmatrix}
\end{equation*}

Dada la siguiente transformación biyectiva:
\begin{multline*}
    \Vec{T}: (0;+\infty) \times (\alpha;\alpha+2\pi) \longrightarrow \setR^2 - L_\alpha \tq
    \\
    \Vec{T}(\rho,\theta) = \begin{bmatrix} \rho \, \cos(\theta) & \rho \, \sin(\theta) \end{bmatrix}
\end{multline*}

Se tiene que:
\begin{equation*}
    \operatorname{det} \big( J \Vec{T}(\rho,\theta) \big) = \rho \, \cos^2(\theta) + \rho \, \sin^2(\theta) = \rho \neq 0
\end{equation*}

Por lo tanto, al aplicar la transformación queda:
\begin{multline*}
    \Vec{F} \Big( \Vec{T}(\rho,\theta) \Big) =
    \\
    = \begin{bmatrix} u \Big( \rho \, \cos{(\theta)} , \rho \, \sin(\theta) \Big) & v \Big( \rho \, \cos(\theta) , \rho \, \sin(\theta) \Big) \end{bmatrix}
\end{multline*}

Y aplicando la regla de la cadena:
\begin{equation*}
    J \Vec{F} \Big( \Vec{T}(\rho,\theta) \Big) = J_{xy} \Vec{F} \Big( \Vec{T}(\rho,\theta) \Big) \cdot J_{\rho\theta} \Vec{T}(\rho,\theta)
\end{equation*}

Esto, representado de forma matricial, es:
\begin{equation*}
    \scale{0.9}{
    \begin{pmatrix}
        \dfrac{\partial u}{\partial \rho} & \dfrac{\partial u}{\partial \theta}
        \\[1em]
        \dfrac{\partial v}{\partial \rho} & \dfrac{\partial v}{\partial \theta}
    \end{pmatrix}_{\Vec{T}(\rho,\theta)}
    =
    \begin{pmatrix}
        \dfrac{\partial u}{\partial x} & \dfrac{\partial u}{\partial y}
        \\[1em]
        \dfrac{\partial v}{\partial x} & \dfrac{\partial v}{\partial y}
    \end{pmatrix}_{\Vec{T}(\rho,\theta)}
    %
    \begin{pmatrix}
        \dfrac{\partial T_1}{\partial \rho} & \dfrac{\partial T_1}{\partial \theta}
        \\[1em]
        \dfrac{\partial T_2}{\partial \rho} & \dfrac{\partial T_2}{\partial \theta}
    \end{pmatrix}_{(\rho,\theta)}
    }
\end{equation*}

O bien, representado en un sistema de ecuaciones:
\begin{equation*}
    \left\{
    \begin{aligned}
        \frac{\partial u}{\partial \rho} &=
        \frac{\partial u}{\partial x} \cdot \frac{\partial T_1}{\partial \rho} + \frac{\partial u}{\partial y} \cdot \frac{\partial T_2}{\partial \rho}
        \\[1ex]
        \frac{\partial u}{\partial \theta} &=
        \frac{\partial u}{\partial x} \cdot \frac{\partial T_1}{\partial \theta} + \frac{\partial u}{\partial y} \cdot \frac{\partial T_2}{\partial \theta}
        \\[1ex]
        \frac{\partial v}{\partial \rho} &=
        \frac{\partial v}{\partial x} \cdot \frac{\partial T_1}{\partial \rho} + \frac{\partial v}{\partial y} \cdot \frac{\partial T_2}{\partial \rho}
        \\[1ex]
        \frac{\partial v}{\partial \theta} &=
        \frac{\partial v}{\partial x} \cdot \frac{\partial T_1}{\partial \theta} + \frac{\partial v}{\partial y} \cdot \frac{\partial T_2}{\partial \theta}
    \end{aligned}
    \right.
\end{equation*}

Resolviendo las derivadas de la transformación, que son conocidas, se tiene:
\begin{equation*}
    \left\{
    \begin{aligned}
        \frac{\partial u}{\partial \rho} &= \frac{\partial u}{\partial x} \cos(\theta) + \frac{\partial u}{\partial y} \sin(\theta)
        \\[1ex]
        \frac{\partial u}{\partial \theta} &= -  \frac{\partial u}{\partial x} \rho \sin(\theta) + \frac{\partial u}{\partial y} \rho \cos(\theta)
        \\[1ex]
        \frac{\partial v}{\partial \rho} &= \frac{\partial v}{\partial x} \cos(\theta) + \frac{\partial v}{\partial y} \sin(\theta)
        \\[1ex]
        \frac{\partial v}{\partial \theta} &= - \frac{\partial v}{\partial x} \rho \sin(\theta) + \frac{\partial v}{\partial y} \rho \cos(\theta)
    \end{aligned}
    \right.
\end{equation*}

Aplicando las ecuaciones de Cauchy-Riemann (Def. \ref{defn:CRequations}) a las dos primeras ecuaciones del sistema anterior, se tiene:

\begin{equation*}
    \left\{
    \begin{aligned}
        \frac{\partial u}{\partial \rho} &= \frac{\partial v}{\partial y} \cos(\theta) - \frac{\partial v}{\partial x} \sin(\theta)
        \\[1ex]
        \frac{\partial u}{\partial \theta} &= -  \frac{\partial v}{\partial y} \rho \sin(\theta) - \frac{\partial v}{\partial x} \rho \cos(\theta)
        \\[1ex]
        \frac{\partial v}{\partial \rho} &= \frac{\partial v}{\partial x} \cos(\theta) + \frac{\partial v}{\partial y} \sin(\theta)
        \\[1ex]
        \frac{\partial v}{\partial \theta} &= -  \frac{\partial v}{\partial x} \rho \sin(\theta) + \frac{\partial v}{\partial y} \rho \cos(\theta)
    \end{aligned}
    \right.
\end{equation*}

Para finalmente, al sacar factor común $\rho$:
\begin{equation*}
    \left\{
    \begin{aligned}
        \frac{\partial u}{\partial \rho} &= \frac{\partial v}{\partial y} \cos(\theta) - \frac{\partial v}{\partial x} \sin(\theta)
        \\[1ex]
        \frac{\partial u}{\partial \theta} &= -\rho \left(  \frac{\partial v}{\partial y} \sin(\theta) + \frac{\partial v}{\partial x} \cos(\theta) \right)
        \\[1ex]
        \frac{\partial v}{\partial \rho} &= \frac{\partial v}{\partial x} \cos(\theta) + \frac{\partial v}{\partial y} \sin(\theta)
        \\[1ex]
        \frac{\partial v}{\partial \theta} &= \rho \left( -  \frac{\partial v}{\partial x} \sin(\theta) + \frac{\partial v}{\partial y} \cos(\theta) \right)
    \end{aligned}
    \right.
\end{equation*}

Establecer una relación para las derivadas parciales en coordenadas polares:

\begin{mdframed}[style=DefinitionFrame]
    \begin{defn}
        \label{defn:CRpolarEquations}
    \end{defn}
    \cusTi{Ecuaciones de C-R en polares}
    \cusTe{Si una función es derivable entonces se cumple que:}
    \begin{align*}
        \frac{\partial}{\partial \rho} u(\rho,\theta) &= \frac{\partial}{\partial \theta}  v(\rho,\theta) \left( \frac{1}{\rho} \right)
        \\[1em]
        \frac{\partial}{\partial \rho} v(\rho,\theta) = - \frac{\partial}{\partial \theta} u(\rho,\theta) \left( \frac{1}{\rho} \right)
    \end{align*}
\end{mdframed}


\subsection{Implicaciones geométricas}

Dado que:
\begin{equation*}
    \frac{\dif}{\dif z} f(z) \neq 0 \Rightarrow  J \Vec{F} \neq \Vec{0} \Rightarrow \grad u \neq \Vec{0} \land \grad v \neq \Vec{0}
\end{equation*}

Donde:
\begin{equation*}
    \left\{
    \begin{aligned}
        \grad u(x,y) &= \frac{\partial}{\partial x} u(x,y) + \iu \frac{\partial}{\partial y} u(x,y)
        \\[1ex]
        \grad v(x,y) &= \frac{\partial}{\partial x} v(x,y) + \iu \frac{\partial}{\partial y} v(x,y)
    \end{aligned}
    \right.
\end{equation*}

De manera que al aplicar las ecuaciones de Cauchy-Riemann (Def. \ref{defn:CRequations}) sobre el sistema anterior se tiene:
\begin{align*}
    \grad v(x,y) &= \frac{\partial}{\partial x} v(x,y) + \iu \frac{\partial}{\partial y} v(x,y)
    \\
    &= - \frac{\partial}{\partial y} u(x,y) + \iu \frac{\partial}{\partial x} u(x,y)
    \\[1ex]
    \grad v(x,y) &= \iu \left( \frac{\partial}{\partial x} u(x,y) + \iu \frac{\partial}{\partial y} u(x,y) \right)
\end{align*}

Por lo tanto:
\begin{equation*}
    \iu \grad u(x,y) = \grad v(x,y) \enspace \textrm{siendo} \enspace \norm{\grad u(x,y)} = \norm{\grad v(x,y)}
\end{equation*}


\section{Derivación en polares}

Aplicando una transformación biyectiva $(T)$ al campo vectorial $(\Vec{F})$ asociado a una función compleja $(f)$, al calcular la matriz de derivadas, o matriz Jacobiana $(J)$, se puede deducir una fórmula para calcular la derivada a partir de las coordenadas polares.

Dada la siguiente función compleja y su campo vectorial asociado:
\begin{equation*}
    f(z) = u(z) + \iu v(z) \iff \Vec{F} = \begin{bmatrix} u(x,y) & v(x,y) \end{bmatrix}
\end{equation*}

Dada la siguiente transformación biyectiva:
\begin{multline*}
    \Vec{T}: (0;+\infty) \times (\alpha;\alpha+2\pi) \longrightarrow \setR^2 - L_\alpha \tq
    \\
    \Vec{T}(\rho,\theta) = \begin{bmatrix} \rho \, \cos(\theta) & \rho \, \sin(\theta) \end{bmatrix}
\end{multline*}

Se tiene que:
\begin{equation*}
    \operatorname{det} \big( J \Vec{T}(\rho,\theta) \big) = \rho \, \cos^2(\theta) + \rho \, \sin^2(\theta) = \rho \neq 0
\end{equation*}

Por lo tanto, al aplicar la transformación queda:
\begin{multline*}
    \Vec{F} \Big( \Vec{T}(\rho,\theta) \Big) =
    \\
    = \begin{bmatrix} u \Big( \rho \, \cos{(\theta)} , \rho \, \sin(\theta) \Big) & v \Big( \rho \, \cos(\theta) , \rho \, \sin(\theta) \Big) \end{bmatrix}
\end{multline*}

Y aplicando la regla de la cadena:
\begin{equation*}
    J \Vec{F} \Big( \Vec{T}(\rho,\theta) \Big) = J_{xy} \Vec{F} \Big( \Vec{T}(\rho,\theta) \Big) \cdot J_{\rho\theta} \Vec{T}(\rho,\theta)
\end{equation*}

Esto, representado de forma matricial donde las filas de las matrices se notan con gradientes, es:
\begin{equation*}
    \begin{bmatrix}
        \grad_{\rho\theta} u
        \\
        \grad_{\rho\theta} v
    \end{bmatrix}_{\Vec{T}(\rho,\theta)}
    =
    \begin{bmatrix}
        \grad_{xy} u
        \\
        \grad_{xy} v
    \end{bmatrix}_{\Vec{T}(\rho,\theta)}
    %
    \begin{bmatrix}
        J_{\rho\theta} \Vec{T}
    \end{bmatrix}_{(\rho,\theta)}
\end{equation*}

Luego, tomando solo uno de los gradientes:
\begin{equation*}
    \begin{bmatrix}
        \grad_{\rho\theta} u
    \end{bmatrix}_{\Vec{T}(\rho,\theta)}
    =
    \begin{bmatrix}
        \grad_{xy} u
    \end{bmatrix}_{\Vec{T}(\rho,\theta)}
    %
    \begin{bmatrix}
        J_{\rho\theta} \Vec{T}
    \end{bmatrix}_{(\rho,\theta)}
\end{equation*}

Se puede calcular el gradiente $(\grad_{xy})$ multiplicando por la matriz inversa del jacobiano $(J^{-1})$:
\begin{gather*}
    \begin{bmatrix}
        J_{\rho\theta} \Vec{T}
    \end{bmatrix}_{(\rho,\theta)}^{-1}
    %
    \begin{bmatrix}
        \grad_{\rho\theta} u
    \end{bmatrix}_{\Vec{T}(\rho,\theta)}^t
    =
    \begin{bmatrix}
        \grad_{xy} u
    \end{bmatrix}_{\Vec{T}(\rho,\theta)}^t
    \\[1em]
    \begin{bmatrix}
        \cos(\theta) & -\dfrac{\sin(\theta)}{\rho}
        \\[1em]
        \sin(\theta) & \dfrac{\cos(\theta)}{\rho}
    \end{bmatrix}
    %
    \begin{bmatrix}
        \grad_{\rho\theta} u
    \end{bmatrix}_{\Vec{T}(\rho,\theta)}^t
    =
    \begin{bmatrix}
        \grad_{xy} u
    \end{bmatrix}_{\Vec{T}(\rho,\theta)}^t
\end{gather*}

\begin{multline*}
    \frac{\partial}{\partial x} u \Big( \Vec{T}(\rho,\theta) \Big) =
    \\
    = \frac{\partial}{\partial \rho} u \Big( \Vec{T}(\rho,\theta) \Big) \cos(\theta) - \frac{\partial}{\partial \theta} u \Big( \Vec{T}(\rho,\theta) \Big) \frac{\sin(\theta)}{\rho}
\end{multline*}

\begin{multline*}
    \frac{\partial}{\partial y} u \Big( \Vec{T}(\rho,\theta) \Big) =
    \\
    = \frac{\partial}{\partial \rho} u \Big( \Vec{T}(\rho,\theta) \Big) \sin(\theta) + \frac{\partial}{\partial \theta} u \Big( \Vec{T}(\rho,\theta) \Big) \frac{\cos(\theta)}{\rho}
\end{multline*}

Por otro lado, aplicando las ecuaciones de Cauchy-Riemann (Def. \ref{defn:CRequations}) en la propiedad \ref{prop:CRtheorem}, se tiene que:
\begin{equation*}
    \frac{\dif}{\dif z} f(z) = \frac{\partial}{\partial x} u(x,y) - \iu \frac{\partial}{\partial y} u(x,y)
\end{equation*}

Por lo que, reemplazando las ecuaciones del sistema en la ecuación anterior se tiene:
\begin{multline*}
    \frac{\dif}{\dif z} f(z) = \left( \frac{\partial u}{\partial \rho} \cos(\theta) - \frac{\partial u}{\partial \theta} \frac{\sin(\theta)}{\rho} \right) +
    \\
    - \iu \left( \frac{\partial u}{\partial \rho} \sin(\theta) + \frac{\partial u}{\partial \theta} \frac{\cos(\theta)}{\rho} \right)
\end{multline*}

Sacando factor común en grupos:
\begin{multline*}
    \frac{\dif}{\dif z} f(z) = \frac{\partial u}{\partial \rho} \Big( \cos(\theta) - \iu \sin(\theta) \Big) +
    \\
    - \frac{1}{\rho} \frac{\partial u}{\partial \theta} \Big( \sin(\theta) + \iu \cos(\theta) \Big)
\end{multline*}

Aplicando las ecuaciones de Cauchy-Riemann en coordenadas polares (Def. \ref{defn:CRpolarEquations}) se tiene:
\begin{multline*}
    \frac{\dif}{\dif z} f(z) = \frac{\partial u}{\partial \rho} \Big( \cos(\theta) - \iu \sin(\theta) \Big) +
    \\
    + \frac{\partial v}{\partial \rho} \Big( \sin(\theta) + \iu \cos(\theta) \Big)
\end{multline*}

Multiplicando y dividiendo por $\iu$ se tiene:
\begin{multline*}
    \frac{\dif}{\dif z} f(z) = \frac{\partial u}{\partial \rho} \Big( \cos(\theta) - \iu \sin(\theta) \Big) +
    \\
    + \iu \frac{\partial v}{\partial \rho} \Big( \cos(\theta) - \iu \sin(\theta) \Big)
\end{multline*}
\begin{gather*}
    \dfrac{\dif}{\dif z} f(z) = \left( \frac{\partial u}{\partial \rho} + \iu \frac{\partial v}{\partial \rho} \right) \Big( \cos(\theta) - \iu \sin(\theta) \Big)
    \\[1em]
    \frac{\dif}{\dif z} f(z) = \left( \frac{\partial}{\partial \rho} u \big( \Vec{T}(\rho,\theta) \big) + \iu \frac{\partial}{\partial \rho} v \big( \Vec{T}(\rho,\theta) \big) \right) e^{-\iu\theta}
\end{gather*}

\begin{mdframed}[style=PropertyFrame]
    \begin{prop}
    \end{prop}
    \begin{equation*}
        \frac{\dif}{\dif z} f(z) = \left[ \frac{\partial}{\partial \rho} u \big( \Vec{T}(\rho,\theta) \big) + \iu \frac{\partial}{\partial \rho} v \big( \Vec{T}(\rho,\theta) \big) \right] e^{-\iu\theta}
    \end{equation*}
\end{mdframed}