\chapter{Extremos de funciones reales}

El conjunto de puntos críticos es la unión de los conjuntos de puntos estacionarios y puntos de no diferenciabilidad.

\begin{mdframed}[style=DefinitionFrame]
    \begin{defn}
    \end{defn}
    \cusTi{Puntos críticos}
    \begin{multline*}
        A = \inBraces{ \Vec{x} \in \setR^\nth \tq \grad f(\Vec{x}) = \Vec{0} }
        \\
        \cup \inBraces{ \Vec{x} \in \setR^\nth \tq f(\Vec{x}) \hspace{1ex} \textrm{no es diferenciable} }
    \end{multline*}
\end{mdframed}

Los puntos críticos son los elementos del dominio que pueden ser un máximo, un mínimo o un punto de ensilladura.

\begin{mdframed}[style=PropertyFrame]
    \begin{prop}
    \end{prop}
    \cusTi{Puntos críticos}
    \begin{gather*}
        \Vec{x}_0 \hspace{1ex} \textrm{es extremo} \lor \Vec{x}_0 \hspace{1ex} \textrm{es punto de ensilladura}
        \\
        \Rightarrow \Vec{x}_0 \hspace{1ex} \textrm{es un punto crítico.}
    \end{gather*}
\end{mdframed}

Un extremo local o relativo evaluado en la función toma el valor más alto o más bajo de la imagen, en el entorno del punto.
Un extremo absoluto o global es un máximo o mínimo no solo para su entorno, sino también para todos los valores que tome la función en el resto del dominio.

\begin{itemize}
    \item Mínimo local: $f(\Vec{x}_0)<f(\Vec{x}) \quad \forall \hspace{1ex} \Vec{x} \in B(\Vec{x}_0, \delta)$

    \item Máximo local: $f(\Vec{x}_0)>f(\Vec{x}) \quad \forall \hspace{1ex} \Vec{x} \in B(\Vec{x}_0, \delta)$

    \item Mínimo absoluto: $f(\Vec{x}_0)<f(\Vec{x}) \quad \forall \hspace{1ex} \Vec{x} \in \operatorname{Dn}(f)$

    \item Máximo absoluto: $f(\Vec{x}_0)>f(\Vec{x}) \quad \forall \hspace{1ex} \Vec{x} \in \operatorname{Dn}(f)$
\end{itemize}

Si $\Vec{x}_0 \in \operatorname{Dn}(f)$ es un punto crítico y no es un extremo, entonces es un punto de ensilladura.

Si trasladamos la gráfica de manera que en el punto en cuestión la imagen sea nula, las secciones de la función en un punto silla son, o bien funciones cúbicas donde la función cambia de signo, o bien parábolas donde en una dirección la parábola es cóncava hacia arriba y en otra hacia abajo.


\section{Método de Sylvester}

Este método para clasificar puntos críticos conciste en calcular el determinante de la matriz Hessiana de la función.
\begin{equation*}
    \left\{
    \begin{aligned}
        \operatorname{det}(H) < 0 & \Rightarrow \Vec{x}_0 \hspace{1ex} \textrm{es punto de ensilladura}
        \\
        \operatorname{det}(H) = 0 &\Rightarrow \hspace{1ex} \textrm{el criterio no aplica}
        \\
        \operatorname{det}(H) > 0 &\Rightarrow \Vec{x}_0 \hspace{1ex} \textrm{es extremo}
    \end{aligned}
    \right.
\end{equation*}

En el último caso podemos determinar si el extremo se trata de un máximo o un mínimo estudiando la derivada segunda.
\begin{equation*}
    \left\{
    \begin{aligned}
        \frac{\partial^2}{\partial x_1^2} f(\Vec{x}_0) < 0 &\Rightarrow \Vec{x}_0 \hspace{1ex} \textrm{es máximo}
        \\[1em]
        \frac{\partial^2}{\partial x_1^2} f(\Vec{x}_0) > 0 &\Rightarrow \Vec{x}_0 \hspace{1ex} \textrm{es mínimo}
    \end{aligned}
    \right.
\end{equation*}


\section{Extremos Condicionados}

El método por multiplicadores de Lagrange sirve para buscar extremos de la función para un conjunto restringido.

Dada $f:D_1 \subseteq \setR^\nth \longrightarrow \setR$ restringida a $g:D_2 \subseteq \setR^\nth \longrightarrow \setR \tq g(\Vec{x})=0 \land \grad g(\Vec{x}) \neq \Vec{0}$ se puede construir la función $L$ que verifica:
\begin{equation*}
    L:D_3 \subseteq \setR^{\nth+1} \longrightarrow \setR \tq L(\Vec{x}, \lambda) = f(\Vec{x}) - \lambda g(\Vec{x})
\end{equation*}

Siendo los candidatos a extremos o puntos de ensilladura de $f$ los mismos puntos críticos de $L$.
Por lo tanto, si $f$ y $g$ son campos escalares de dos variables, se tiene que:
\begin{gather*}
    \grad L(x,y,\lambda)=0 \iff
    \\[1em]
    \left\{
    \begin{aligned}
        \frac{\partial}{\partial x} f(x,y) - \lambda \frac{\partial}{\partial x} g(x,y) &= 0
        \\[1ex]
        \frac{\partial}{\partial y} f(x,y) - \lambda \frac{\partial}{\partial y} g(x,y) &= 0
        \\[1ex]
        \frac{\partial}{\partial \lambda} f(x,y) - \lambda \frac{\partial}{\partial \lambda} g(x,y) &= 0
    \end{aligned}
    \right.
    \\[1em]
    \left\{
    \begin{aligned}
        \frac{\partial}{\partial x} f(x,y) &= \lambda \frac{\partial}{\partial x} g(x,y)
        \\[1ex]
        \frac{\partial}{\partial y} f(x,y) &= \lambda \frac{\partial}{\partial y} g(x,y)
        \\[1ex]
        g(x,y) &= 0
    \end{aligned}
    \right.
\end{gather*}