\chapter{Números complejos}

Los números complejos surgen de la necesidad de inventar una solución a la ecuación $x^2=-1$.
Es por esto que se define el número imaginario $\iu$ de manera tal que $\iu^2=-1$.

El número $\iu$ pertenece al conjunto $\setI$ de números imaginarios.
Los números imaginarios se forman multiplicando el número $\iu$ por un número real $y$.

Combinando los números reales $\setR \subset \setC$ con los números imaginarios $\setI \subset \setC$ surge el conjunto de los números complejos $\setC=\setR \times \setI$.

A diferencia de los reales, el conjunto de números complejos no es un conjunto ordenado.
Los números complejos se representan gráficamente en un plano.
El plano complejo tiene en el eje $x$ la conocida recta real mientras que el eje vertical representa los números $\iu \, y$ imaginarios puros.

Los complejos se pueden expresar de varias formas que se relaciónan entre sí según la fórmula de Euler:

\begin{mdframed}[style=PropertyFrame]
    \begin{prop}
        \label{prop:EulerFormula}
    \end{prop}
    \cusTi{Fórmula de Euler}
    \begin{equation*}
        \cos{(\theta)} \pm \iu \sin{(\theta)} = e^{\pm \iu \theta}
    \end{equation*}
\end{mdframed}

\begin{mdframed}[style=PropertyFrame]
    \begin{prop}
    \end{prop}
    \cusTi{Fórmula de Moivre}
    \begin{equation*}
        \cos{(n \theta)} \pm \iu \sin{(n \theta)} = e^{\pm \iu n \theta}
    \end{equation*}
\end{mdframed}

Un número complejo se puede definir como la suma de un número real con un número imaginario.

\begin{mdframed}[style=DefinitionFrame]
    \begin{defn}
        \label{defn:BinomialForm}
    \end{defn}
    \cusTi{Forma binomial}
    \begin{equation*}
        z = x + \iu y
    \end{equation*}
\end{mdframed}

Se puede definir como un par ordenado perteneciente al plano complejo, donde la primer componente es la parte real y la segunda es la parte imaginaria.

\begin{mdframed}[style=DefinitionFrame]
    \begin{defn}
        \label{defn:VectorForm}
    \end{defn}
    \cusTi{Forma vectorial}
    \begin{equation*}
        z = (x,y) = \left[\Re(z),\Im(z)\right] \in \setC
    \end{equation*}
\end{mdframed}

Podemos representar un número complejo de forma trigonométrica como un vector de cierto largo $\rho=\norm{z}$ que forma cierto ángulo $\theta = \artan{(x/y)}$ con el eje real.

\begin{mdframed}[style=DefinitionFrame]
    \begin{defn}
        \label{defn:TrigForm}
    \end{defn}
    \cusTi{Forma polar}
    \begin{equation*}
        z = \rho \, \big( \cos{(\theta)} + \iu \sin{(\theta)} \big)
    \end{equation*}
\end{mdframed}

O bien podemos expresar un complejo de forma exponencial usando la fórmula de Euler.

\begin{mdframed}[style=DefinitionFrame]
    \begin{defn}
        \label{defn:ExpForm}
    \end{defn}
    \cusTi{Forma exponencial}
    \begin{equation*}
        z = \rho \, e^{\iu \theta}
    \end{equation*}
\end{mdframed}


\section{Producto}

El producto de dos números complejos da como resultado otro número complejo.
Si bien los numeros complejos pueden ser expresados vectorialmente, el producto no es el producto interno de vectores convencionales.

El producto se define como la multiplicación de números complejos expresados en forma binomial.

\begin{mdframed}[style=DefinitionFrame]
    \begin{defn}
    \end{defn}
    \cusTi{Producto}
    \begin{equation*}
        (x_1 + \iu y_1) \cdot (x_2 + \iu y_2) = (x_1x_2-y_1y_2) + \iu (y_1x_2+x_1y_2)
    \end{equation*}
\end{mdframed}

\begin{mdframed}[style=PropertyFrame]
    \begin{prop}
    \end{prop}
    Para $z_3=z_1 \cdot z_2 = \norm{z_3} e^{\iu \arg(z_3) } $ se tiene que:
    \begin{gather*}
        \norm{z_3} = \norm{z_1} \cdot \norm{z_2}
        \\[1ex]
        \arg(z_3) = \arg(z_1) + \arg(z_2)
    \end{gather*}
\end{mdframed}

Al multiplicar un número por su inverso, se obtiene la unidad.

\begin{mdframed}[style=DefinitionFrame]
    \begin{defn}
        \label{defn:Inverse}
    \end{defn}
    \cusTi{Inverso multiplicativo}
    \begin{equation*}
        z^{-1}=\frac{1}{z} \iff z \cdot z^{-1} = 1
    \end{equation*}
\end{mdframed}

\begin{mdframed}[style=PropertyFrame]
    \begin{prop}
    \end{prop}
    \begin{equation*}
        z^{-1}=\dfrac{\conj{z}}{\norm{z}^2}
    \end{equation*}
\end{mdframed}

\concept{Demostración por definición}

Dado $x + \iu y$ y su inverso $u + \iu v$, por definición su producto es $1 + \iu 0$ o bien, expresado en forma vectorial:
\begin{gather*}
    (x,y) \cdot (u,v) = (1,0)
    \\
    (xu-yv,yu+xv) = (1,0)
\end{gather*}

Obteniendo un sistema de ecuaciones del que es posible despejar las componentes del vector que representan al inverso.
\begin{gather*}
    \left\{
    \begin{aligned}
        1 &= xu-yv
        \\
        0 &= yu+xv
    \end{aligned}
    \right.
    \\
    u = -\frac{xv}{y} \Rightarrow 1= -\frac{x^2v}{y}-yv = -v\left( \dfrac{x^2}{y}+y \right)
    \\
    v=\frac{-1}{\frac{x^2}{y}+y}=\dfrac{-y}{\frac{x^2y}{y}+y^2}
\end{gather*}

Luego:
\begin{gather*}
    \left\{
    \begin{aligned}
        u &= \frac{x}{x^2+y^2}
        \\[1ex]
        v &= \frac{-y}{x^2+y^2}
    \end{aligned}
    \right.
    \\
    z^{-1} = \dfrac{x}{x^2+y^2}-\dfrac{\iu y}{x^2+y^2}
\end{gather*}

\concept{Demostración por el conjugado}

Esta demostración es más rápida pero en ella se utilizan elementos todavía no definidos.
Ver primero la definición \ref{defn:conjugate}, y las propiedades \ref{prop:eAbsoluteValue} y \ref{prop:|z|^2=z.z*}.
\begin{align*}
    z \cdot \conj{z} &= \norm{z}^2
    \\
    \frac{z \cdot \conj{z}}{\norm{z}^2} &= 1
    \\
    \frac{\conj{z}}{\norm{z}^2} &= \dfrac{1}{z}=z^{-1}
\end{align*}


\section{Módulo}

El módulo de un número complejo es el largo del vector que lo representaría si este se graficase en el plano complejo.

\begin{mdframed}[style=DefinitionFrame]
    \begin{defn}
    \end{defn}
    \cusTi{Módulo}
    \begin{equation*}
        \norm{z} = \nnorm{(x,y)} = \sqrt{x^2+y^2}
    \end{equation*}
\end{mdframed}

El módulo de la resta de dos números complejos es la distancia que hay entre esos dos puntos:

\begin{mdframed}[style=PropertyFrame]
    \begin{prop}
        \label{prop:distance}
    \end{prop}
    \begin{equation*}
        \operatorname{dist}(z_1,z_2)=\norm{z_1-z_2}
    \end{equation*}
\end{mdframed}

El módulo del número $e^{\iu}$ se puede expresar como:
\begin{equation*}
    \norm{e^{\iu \theta}} = \norm{\cos(\theta) + \iu \sin(\theta)} = \sqrt{\cos^2(\theta) + \sin^2(\theta)}
\end{equation*}

Con lo cual, $e^{\iu}$ tiene módulo unitario.

\begin{mdframed}[style=PropertyFrame]
    \begin{prop}
        \label{prop:eAbsoluteValue}
    \end{prop}
    \begin{equation*}
        \norm{e^{\iu \theta}} = 1
    \end{equation*}
\end{mdframed}

El módulo de un número es mayor que sus componentes:

\begin{mdframed}[style=PropertyFrame]
    \begin{prop}
        \label{prop:zAbsoluteValue}
    \end{prop}
    \begin{equation*}
        \Re(z) \leq \norm{\Re(z)} \leq \norm{z} \geq \norm{\Im(z)} \geq \Im(z)
    \end{equation*}
\end{mdframed}

\begin{mdframed}[style=PropertyFrame]
    \begin{prop}
        \label{prop:triangleInequality}
    \end{prop}
    \cusTi{Desigualdad triangular}
    \begin{equation*}
        \big| \norm{z_1}-\norm{z_2} \big| \leq \big| z_1 \pm z_2 \big|\leq \norm{z_1} + \norm{z_2}
    \end{equation*}
\end{mdframed}

\concept{Demostración}

La parte derecha de la desigualdad es evidente geométricamente, y la otra se demuestra como un corolario de la primera.

Al plantear los dos siguientes casos, quedan demostradas las dos situaciones que contempla el módulo en la desigualdad izquierda de la inecuación:

\begin{itemize}
    \item Caso $\norm{z_2} \leq \norm{z_1}$
    \begin{gather*}
        \norm{z_1} = \norm{(z_1+z_2)-z_2} \leq \norm{z_1+z_2}+\norm{-z_2}
        \\
        \norm{z_1} - \norm{-z_2} \leq \norm{z_1+z_2}
        \\
        \norm{z_1} - \norm{z_2} \leq \norm{z_1+z_2}
    \end{gather*}

    \item Caso $\norm{z_1} \leq \norm{z_2}$
    \begin{gather*}
        \norm{z_2} = \norm{(z_2+z_1)-z_1} \leq \norm{z_2+z_1}+\norm{-z_1}
        \\
        \norm{z_2} - \norm{-z_1} \leq \norm{z_2+z_1}
        \\
        \norm{z_2} - \norm{z_1} \leq \norm{z_1+z_2}
    \end{gather*}
\end{itemize}


\section{Conjugado}

El conjugado $(\conj{z})$ de un número $x+iy$ se define cambiando el signo del termino imaginario.

\begin{mdframed}[style=DefinitionFrame]
    \begin{defn}
        \label{defn:conjugate}
    \end{defn}
    \cusTi{Conjugado}
    \begin{equation*}
        \conj{z} = x - \iu y
    \end{equation*}
\end{mdframed}

El conjugado de un número es simétrico a este con respecto del eje real, por lo tanto ambos tienen el mismo módulo:

\begin{mdframed}[style=PropertyFrame]
    \begin{prop}
    \end{prop}
    \begin{equation*}
        \norm{z} = \norm{\conj{z}}
    \end{equation*}
\end{mdframed}

El conjugado de la suma de dos números es la suma de los conjuados, y análogamente para el producto:

\begin{mdframed}[style=PropertyFrame]
    \begin{prop}
        \label{prop:zw*=(z*w)*}
    \end{prop}
    \begin{gather*}
        \conj{z_1 \pm z_2} = \conj{z_1} \pm \conj{z_2}
        \\
        \conj{z_1 \cdot z_2} = \conj{z_1} \cdot \conj{z_2}
    \end{gather*}
\end{mdframed}

Sumando el conjugado de un número con este, se obtiene el doble de la parte real y análogamente para la parte imaginaria al hacer la diferencia.
\begin{gather*}
        z+\conj{z} = 2x
        \\
        z-\conj{z} = 2 \iu y
    \end{gather*}

Quedando definida la siguiente propiedad.

\begin{mdframed}[style=PropertyFrame]
    \begin{prop}
        \label{prop:ReIm}
    \end{prop}
    \begin{align*}
        \Re (z) &= \frac{z+\conj{z}}{2}
        \\[1ex]
        \Im (z) &= \frac{z-\conj{z}}{2 \iu}
    \end{align*}
\end{mdframed}

Multiplicando un complejo por su conjugado queda:
\begin{equation*}
    z \cdot \conj{z} = x^2+y^2
\end{equation*}

Que es el módulo al cuadrado.

\begin{mdframed}[style=PropertyFrame]
    \begin{prop}
        \label{prop:|z|^2=z.z*}
    \end{prop}
    \begin{equation*}
        z \cdot \conj{z} = \norm{z}^2
    \end{equation*}
\end{mdframed}

Si un número complejo es igual a su conjugado, entonces se trata de un número real.

\begin{mdframed}[style=PropertyFrame]
    \begin{prop}
    \end{prop}
    \begin{equation*}
        z = \conj{z} \iff z \in \setR
    \end{equation*}
\end{mdframed}

Si el cuadrado de un número es igual al cuadrado de su conjugado, entonces se trata o bien de un real o bien de un imaginario puro.

\begin{mdframed}[style=PropertyFrame]
    \begin{prop}
    \end{prop}
    \begin{equation*}
        z^2=\conj{z}^2 \iff z \in \setR \enspace \lor \enspace z \in \setI
    \end{equation*}
\end{mdframed}


\section{Potenciación}

Elevar un complejo a la n-ésima potencia se define, para $\nth \in \setZ$, a partir de la forma exponencial de la siguiente manera.

\begin{mdframed}[style=DefinitionFrame]
    \begin{defn}
        \label{defn:nPower}
    \end{defn}
    \cusTi{Potencia n-ésima}
    \begin{equation*}
        z^\nth = \rho^\nth e^{\iu \nth \theta}
    \end{equation*}
\end{mdframed}

El conjugado de un número complejo elevado es igual a la potencia del conjugado.

\begin{mdframed}[style=PropertyFrame]
    \begin{prop}
    \end{prop}
    \begin{equation*}
        \conj{z^\nth} = \left( \conj{z} \right)^\nth
    \end{equation*}
\end{mdframed}

Al igual que podemos calcular las raíces de los polinomios reales de segundo grado, podemos deducir la fórmula resolvente para un polinomio complejo con coeficientes $a,b,c \in \setR$ genéricos:
\begin{equation*}
    az^2 + bz + c = 0
\end{equation*}

En primer lugar, se completan cuadrados:
\begin{gather*}
    z^2 + \left(\frac{b}{a} \right) z + \frac{c}{a} = 0
    \\
    \left( z+\frac{b}{2a} \right)^2 + \left( \frac{c}{a}-\frac{b^2}{4a^2} \right) = 0
    \\
    \left( z+\frac{b}{2a} \right)^2 - \left( \frac{b^2}{4a^2} - \frac{c}{a} \right) = 0
    \\
    \norm{z+\frac{b}{2a}} = \sqrt{\frac{b^2}{4a^2} - \frac{c}{a}}
    = \sqrt{\frac{a \, b^2 - 4 \, a^2 \, c}{4 \, a^3}}
    = \frac{\sqrt{b^2 - 4ac}}{\norm{2a}}
    \\
    z_{1,2} = \frac{-b \pm \sqrt{b^2-4ac}}{2a}
\end{gather*}

Donde:
\begin{equation*}
    \textrm{Si} \enspace \frac{b^2}{4a^2} - \frac{c}{a} \geq 0 \Rightarrow b^2 \geq 4ac \Rightarrow z \in \setR
\end{equation*}

Observar que en el despeje algebráico queda una raíz cuadrada.
De tratarse solamente de números reales tendría un radicando positivo.
Pero si proponemos una solución compleja, puede tener un radicando negativo.
Para expresar esto, vamos a valernos de la definición del número $\iu$ partiendo de la segunda ecuación:
\begin{gather*}
    \left( z+\frac{b}{2a} \right)^2 + \left( \frac{c}{a} - \frac{b^2}{4a^2} \right) = 0
    \\
    \left( z+\frac{b}{2a} \right)^2 - \iu^2 \left( \frac{c}{a}-\frac{b^2}{4a^2} \right) = 0
    \\
    \left( z+\frac{b}{2a} \right)^2 - \left( \iu \sqrt{\frac{4ac-b^2}{4a^2}} \right)^2 = 0
    \\
    \scale{0.98}
    {
    \left( z+\frac{b}{2a} + \iu \sqrt{\frac{4ac-b^2}{4a^2}} \right) \left( z+\frac{b}{2a} - \iu \sqrt{\frac{4ac-b^2}{4a^2}} \right) = 0
    }
\end{gather*}

Y para que el producto anterior sea igual a cero se tiene, o bien:
\begin{equation*}
    z + \frac{b}{2a} + \frac{\iu \sqrt{4ac-b^2}}{2a} = 0
\end{equation*}

O bien:
\begin{equation*}
    z + \frac{b}{2a} - \frac{\iu \sqrt{4ac-b^2}}{2a} = 0
\end{equation*}

Infiriendo así la fórmula resolvente para polinomios complejos de segundo grado:

\begin{mdframed}[style=PropertyFrame]
    \begin{prop}
    \end{prop}
    \cusTi{Fórmula resolvente}
    \begin{equation*}
        z_{1,2} = \frac{-b \pm \iu \sqrt{4ac-b^2}}{2a}
    \end{equation*}
\end{mdframed}

Donde:
\begin{equation*}
    \textrm{Si} \enspace \frac{b^2}{4a^2} - \frac{c}{a} \leq 0 \Rightarrow b^2 \leq 4ac \Rightarrow z \in \setC
\end{equation*}

Así como se pueden buscar las raíces que anulan los polinomios de números reales, se pueden calcular las raíces complejas que, elevadas a la $n$-ésima potencia, dan como resultado cierto $z_0$ conocido.

El conjunto de raíces n-ésimas se define como:
\begin{align*}
    C &= \inBraces{z \in \setC / z^n=z_0}
    \\
    &= z_1,z_2,z_3 \ldots z_\nth
\end{align*}

Aplicando la definición de potencia $n$-ésima (\ref{defn:nPower}) en la parte izquierda de la ecuación y escribiendo la parte derecha usando la forma exponencial, se puede deducir una fórmula para despejar $z$, la incógnita.
\begin{gather*}
    \rho^\nth \cdot e^{\iu \nth \theta} = \rho_0 \cdot e^{\iu \theta_0}
    \\
    \left\{
    \begin{aligned}
        \rho^\nth &= \rho_0
        \\
        \nth \theta &= \theta_0 + 2\kth \pi \quad \textrm{con} \enspace \kth \in \setZ
    \end{aligned}
    \right.
    \\[1ex]
    z_{\kth+1} = \sqrt[\nth]{\rho_0} \enspace \sqrt[\nth]{e^{\iu (\theta_0 + 2 \kth \pi)}} \quad \textrm{con} \enspace \kth \in \{0,1,2 \ldots \nth-1\}
\end{gather*}

Nótese que el subíndice $\kth+1$ de cada raiz es solamente para evitar cometer abuso de notación usando la notación $z_0$ tanto para la primera raiz como para el número al que le estamos calculando las raices.

Se obtiene entonces:

\begin{mdframed}[style=PropertyFrame]
    \begin{prop}
    \end{prop}
    \begin{equation*}
        z_{\kth+1} = \sqrt[\nth]{\rho_0} \enspace e^{\iu \left( \tfrac{\theta_0}{\nth} + \tfrac{2 \kth \pi}{\nth} \right)}
    \end{equation*}
\end{mdframed}

En la notación se suele diferenciar cualquier raiz $n$-ésima $z_{k+1} = z_0^{1/\nth}$ de la única raíz positiva, que recibe el nombre de raíz principal y se denota $z_1=\sqrt[n]{z_0}$.

Se puede observar que la fórmula anterior puede generar $\nth$ valores distintos.
De no acotar los valores que puede tomar $\kth$, la formula daría valores repetidos.
Las raíces $n$-ésimas van a estar distribuidas en el plano complejo formando un polígono de $\nth$ vértices, separados cada uno del otro por el mismo ángulo.

Separando el exponente, cada una de las $\nth$ raíces que arroja la fórmula se puede expresar como un producto en el que uno de los factores siempre va a ser la raíz principal $z_1$.
De esta forma, podemos definir una notación equivalente para el conjunto $C$ de raíces $n$-ésimas:
\begin{equation*}
    C = \inBraces{z_1 , z_1 \, w , z_1 \, w^2 \ldots z_1 \, w^{n-1}}
\end{equation*}

Donde:
\begin{equation*}
    w = e^{\iu \left( \tfrac{2 \pi}{\nth} \right)}
\end{equation*}

Observar que este análisis sirve para estudiar raíces de monomios.
En caso de tratarse de un polinomio de más de un término, no podemos conocer las raíces por definición directamente.

\begin{mdframed}[style=PropertyFrame]
    \begin{prop}
    \end{prop}
    \begin{equation*}
        \left( 1 + z + z^2 + z^3 \ldots + z^{\nth-1} \right) \left( 1 - z \right) = 1 - z^\nth
    \end{equation*}
\end{mdframed}


\section{Geometría de conjuntos}

Podemos representar un conjunto de números complejos como un conjunto de puntos en el plano complejo.
Algunas secciones cónicas, en vez de ser expresadas con sus ecuaciones tradicionales en función de las partes real $(x)$ e imaginaria $(y)$, se pueden expresar en función de los puntos del plano complejo $(z)$ según la propiedad \ref{prop:distance} del módulo.
Es posible verificar que los complejos expresados en su forma binomial verifican también las ecuaciones tradicionales de las cónicas.

\begin{mdframed}[style=DefinitionFrame]
    \begin{defn}
    \end{defn}
    \cusTi{Línea}
    \cusTe{Conjunto de puntos $z \in \setC$ que, por simetría, están a igual distancia de dos puntos $z_1$ y $z_2$, verificando:}
    \begin{equation*}
        \norm{z-z_1} = \norm{z-z_2}
    \end{equation*}
\end{mdframed}

\begin{center}
    \def\svgwidth{0.6\linewidth}
    \input{./images/calc-geom-1.pdf_tex}
\end{center}

\begin{mdframed}[style=DefinitionFrame]
    \begin{defn}
    \end{defn}
    \cusTi{Circunferencia}
    \cusTe{Conjunto de los puntos $z \in \setC$ que están a cierta distancia $R \in \setR$ de un centro $z_0$ en el plano complejo, verificando:}
    \begin{equation*}
        \norm{z-z_0} = R
    \end{equation*}
\end{mdframed}

\begin{center}
    \def\svgwidth{0.6\linewidth}
    \input{./images/calc-geom-2.pdf_tex}
\end{center}

\begin{mdframed}[style=DefinitionFrame]
    \begin{defn}
    \end{defn}
    \cusTi{Elipse}
    \cusTe{Conjunto de puntos $z \in \setC$ que verifican que la suma de las distancias desde los mismos hasta cada uno de dos focos $F_n \in \setC$ vale constantemente $D \in \setR$, verificando:}
    \begin{equation*}
        \norm{z-F_1} + \norm{z-F_2} = D
    \end{equation*}
\end{mdframed}

\begin{center}
    \def\svgwidth{0.6\linewidth}
    \input{./images/calc-geom-3.pdf_tex}
\end{center}