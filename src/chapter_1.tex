\chapter{Topología}

\section{Norma}

\begin{mdframed}[style=DefinitionFrame]
    \begin{defn}
    \end{defn}
    \cusTi{Norma}
    \cusTe{La norma es la distancia entre dos puntos $P$ y $Q$.}
    \begin{equation*}
        \nnorm{P-Q} = \operatorname{dist} (P,Q)
    \end{equation*}
\end{mdframed}

Hay varios tipos de normas, siendo algunas definiciones más precisas que otras con la desventaja de ser computacionalmente menos eficientes.

\begin{mdframed}[style=DefinitionFrame]
    \begin{defn}
    \end{defn}
    \cusTi{Norma 1}
    \begin{equation*}
        \nnorm{\Vec{x}}_1 = \sum_{\ith=1}^\nth \norm{x_\ith}
    \end{equation*}
\end{mdframed}

\begin{mdframed}[style=DefinitionFrame]
    \begin{defn}
        \label{defn:norm2}
    \end{defn}
    \cusTi{Norma 2}
    \begin{equation*}
        \nnorm{\Vec{x}}_2 = \sqrt{\sum_{\ith=1}^\nth x_\ith^2}
    \end{equation*}
\end{mdframed}

\begin{mdframed}[style=DefinitionFrame]
    \begin{defn}
    \end{defn}
    \cusTi{Norma infinito}
    \begin{equation*}
        \nnorm{\Vec{x}}_\infty = \operatorname{max} \big\{ \norm{x_\ith} \big\}
    \end{equation*}
\end{mdframed}


\concept{Propiedades de la norma 2}

\begin{align*}
    \nnorm{\Vec{x}} &= 0 \iff \Vec{x} = 0
    \\[1ex]
    \nnorm{k \, \Vec{x}} &= \norm{k} \cdot \nnorm{\Vec{x}} \quad k\in\setR
    \\[1ex]
    \nnorm{\Vec{x}} &\leq \nnorm{\Vec{x}} + \nnorm{\Vec{y}}
    \\[1ex]
    \nnorm{\Vec{x}+\Vec{y}} &\leq \nnorm{\Vec{x}} + \nnorm{\Vec{y}}
    \\[1ex]
    \norm{\Vec{x} \cdot \Vec{y}} &\leq \nnorm{\Vec{x}} \cdot \nnorm{\Vec{y}}
    \\[1ex]
    \norm{x_\ith} &= \nnorm{\Vec{x}}
\end{align*}


\section{Entornos y bolas}

El entorno o bola de un punto es un conjunto que contiene los puntos más próximos a él.
Se lo llama entorno cuando se trata de un intervalo en una recta, disco cuando se trata de un plano y bola en el espacio o en $\nth$ dimensiones.
Dependiendo de si contiene o no a los puntos del borde o al punto del centro se pueden definir bolas abiertas, cerradas y reducidas.

\begin{mdframed}[style=DefinitionFrame]
    \begin{defn}
    \end{defn}
    \cusTi{Bola cerrada}
    \begin{equation*}
        \Bar{B} (\Vec{x}_0;r) = \{ \Vec{x} \in \setR^n :  \nnorm{\Vec{x}-\Vec{x}_0} \leq r \}
    \end{equation*}
\end{mdframed}

\begin{mdframed}[style=DefinitionFrame]
    \begin{defn}
    \end{defn}
    \cusTi{Bola abierta}
    \begin{equation*}
        B(\Vec{x}_0;r) = \{ \Vec{x} \in \setR^\nth : \nnorm{\Vec{x}-\Vec{x}_0} <r \}
    \end{equation*}
\end{mdframed}

\begin{mdframed}[style=DefinitionFrame]
    \begin{defn}
    \end{defn}
    \cusTi{Bola reducida}
    \begin{equation*}
        B' (\Vec{x}_0;r) = \{ \Vec{x} \in \setR^n :  \nnorm{\Vec{x}-\Vec{x}_0} <r \} - \Vec{x}_0
    \end{equation*}
\end{mdframed}

A continuación se grafican un disco cerrado, uno abierto y uno reducido respectivamente.

\begin{center}
    \def\svgwidth{0.8\linewidth}
    \input{./images/topo-bola.pdf_tex}
\end{center}


\section{Clasificación de puntos}

\begin{mdframed}[style=DefinitionFrame]
    \begin{defn}
    \end{defn}
    \cusTi{Punto interior}
    \begin{equation*}
        \Vec{x}_0 \in A^0 \iff \exists \, r>0 \tq B(\Vec{x}_0;r) \subset A
    \end{equation*}
\end{mdframed}

\begin{mdframed}[style=DefinitionFrame]
    \begin{defn}
    \end{defn}
    \cusTi{Punto exterior}
    \begin{equation*}
        \Vec{x}_0 \in A^c \iff \exists \, r>0 \tq B(\Vec{x}_0;r) \subset A^c
    \end{equation*}
\end{mdframed}

\begin{mdframed}[style=DefinitionFrame]
    \begin{defn}
    \end{defn}
    \cusTi{Punto frontera}
    \begin{multline*}
        \Vec{x}_0 \in \partial A \iff \forall \, r>0 : \\ B(\Vec{x}_0;r) \cap A \neq \setO \quad \land \quad B(\Vec{x}_0;r) \cap A^c \neq \setO
    \end{multline*}
\end{mdframed}

\begin{mdframed}[style=DefinitionFrame]
    \begin{defn}
    \end{defn}
    \cusTi{Punto aislado}
    \begin{equation*}
        \Vec{x}_0 : \exists \, r>0 / \big\{ B(\Vec{x}_0;r) \cap A \big\} = \Vec{x}_0
    \end{equation*}
\end{mdframed}

\begin{mdframed}[style=DefinitionFrame]
    \begin{defn}
        \label{defn:limitPoint}
    \end{defn}
    \cusTi{Punto de acumulación}
    \begin{equation*}
        \Vec{x}_0 \in A' \iff \forall r>0 : \big\{ B'(\Vec{x}_0;r) \cap A \big\} \neq \setO
    \end{equation*}
\end{mdframed}


\section{Clasificación de conjuntos}

\begin{mdframed}[style=DefinitionFrame]
    \begin{defn}
    \end{defn}
    \cusTi{Complemento de un conjunto}
    \begin{equation*}
        A^c = \setR^\nth - A
    \end{equation*}
\end{mdframed}

\begin{mdframed}[style=DefinitionFrame]
    \begin{defn}
    \end{defn}
    \cusTi{Conjunto acotado}
    \begin{equation*}
        A: \exists \, r>0 / A \subset B(0;r)
    \end{equation*}
\end{mdframed}

\begin{mdframed}[style=DefinitionFrame]
    \begin{defn}
    \end{defn}
    \cusTi{Conjunto abierto}
    \begin{equation*}
        A: A=A^0
    \end{equation*}
\end{mdframed}

\begin{mdframed}[style=DefinitionFrame]
    \begin{defn}
    \end{defn}
    \cusTi{Conjunto cerrado}
    \begin{equation*}
        A: \big\{ \partial A \subset A \big\} \lor \big\{ A=\operatorname{cl}(A) \big\}
    \end{equation*}
\end{mdframed}

\begin{mdframed}[style=DefinitionFrame]
    \begin{defn}
    \end{defn}
    \cusTi{Conjunto compacto}
    \cusTe{$A$ es un conjunto compacto si es cerrado y acotado simultáneamente.}
\end{mdframed}

\begin{mdframed}[style=DefinitionFrame]
    \begin{defn}
    \end{defn}
    \cusTi{Conjunto conexo}
    \cusTe{Para todo par de puntos existe al menos una trayectoria que los une.}
\end{mdframed}

\begin{mdframed}[style=DefinitionFrame]
    \begin{defn}
    \end{defn}
    \cusTi{Conjunto simplemente conexo}
    \cusTe{Para todo par de curvas dentro del conjunto que comparten sus extremos existe una deformación tal que es posible convertir una curva en la otra dejando fijos los extremos.}
\end{mdframed}

\begin{mdframed}[style=DefinitionFrame]
    \begin{defn}
    \end{defn}
    \cusTi{Conjunto convexo}
    \cusTe{Para todo par de puntos existe una trayectoria recta que los une.}
\end{mdframed}