\chapter{Funciones reales}

\section{Clasificación de funciones}

Podemos clasificar las funciones según sus conjuntos de partida y llegada.

\begin{mdframed}[style=DefinitionFrame]
    \begin{defn}
    \end{defn}
    \cusTi{Funciones escalares}
    \begin{equation*}
        f:\setR \longrightarrow \setR \tq f(x)=y
    \end{equation*}
\end{mdframed}

\begin{mdframed}[style=DefinitionFrame]
    \begin{defn}
    \end{defn}
    \cusTi{Funciones vectoriales}
    \begin{multline*}
        \Vec{F}:\setR \longrightarrow \setR^\mth \tq
        \\
        \Vec{F}(x) = \begin{bmatrix} F_1(x) & F_2(x) & \dots & F_\mth(x) \end{bmatrix}
    \end{multline*}
\end{mdframed}

\begin{mdframed}[style=DefinitionFrame]
    \begin{defn}
    \end{defn}
    \cusTi{Campos escalares}
    \begin{equation*}
        f:\setR^\nth \longrightarrow \setR \tq f(x_1,x_2 \dots x_\nth)=y
    \end{equation*}
\end{mdframed}

\begin{mdframed}[style=DefinitionFrame]
    \begin{defn}
    \end{defn}
    \cusTi{Campos vectoriales}
    \begin{multline*}
        f:\setR^\nth \longrightarrow \setR^\mth \tq
        \\
        \Vec{F}(\Vec{x}) = \begin{bmatrix} F_1(\Vec{x}) & F_2(\Vec{x}) & \dots & F_\mth(\Vec{x}) \end{bmatrix}
    \end{multline*}
\end{mdframed}


\section{Dominio}

El dominio es un subconjunto del conjunto de partida de una función.
El dominio es el conjunto de elementos que admiten las variables de una función.

Por ejemplo, suponiendo que la función tenga una división, una raíz de orden par o un logaritmo, los elementos excluidos del dominio son aquellos que anulen el denominador, que hagan negativo el radicando o que hagan menor o igual a cero el logaritmo.
Cualquier otra operación estaría permitida, y estos elementos que no presentan inconvenientes están en el dominio de la función.


\section{Conjuntos de nivel, secciones y trazas}

Los conjuntos de nivel, las secciones y las trazas son herramientas útiles para conocer cómo es la gráfica de una función de forma analítica.


\concept{Conjuntos de nivel}

Un conjunto de nivel es un subconjunto del dominio de una función.
Se define fijando la imagen, dándole un valor arbitrario para poder establecer relaciones entre las variables independientes del dominio, restringiéndolo.

\begin{mdframed}[style=DefinitionFrame]
    \begin{defn}
    \end{defn}
    \cusTi{Conjunto de nivel}
    \cusTe{El conjunto de nivel $k$ es aquel que forman todos los vectores del dominio tales que la función de esos vectores tiene como valor $k$.}
    \begin{equation*}
        A_{k} = \big\{ \Vec{x} \in \operatorname{Dn}(f) \tq f(\Vec{x})=k \big\}
    \end{equation*}
\end{mdframed}

Dependiendo de la dimensión del conjunto de salida de $f$, las preimágenes de $k$ van a ser, una curva si $n=2$ o una superficie si $n=3$.
Es decir que los conjuntos de nivel son curvas de nivel si se trata de un campo escalar de 2 variables y superficies de nivel si se trata de un campo escalar de 3 variables.

La gráfica del conjunto de nivel tiene una dimensión menos que la gráfica de la función.
Por este mismo motivo podemos graficar una superficie de nivel de una función cuya gráfica estaría en 4 dimensiones, aunque no podamos graficar la función.

Algebraicamente es posible definir conjuntos de nivel en funciones para $n>3$, pero no podrían graficarse.


\concept{Secciones}

Una sección fija alguna de las variables independientes para conocer qué forma tienen los elementos de la imagen en función de las otras variables que no se fijaron.

Para una función de dos variables, visualmente la gráfica de la función se vería cortada por un plano vertical.


\concept{Trazas}

Una traza es la intersección de la gráfica de una función con un plano cualquiera, en cualquier dirección.


\section{Inyectividad y sobreyectividad}

\begin{mdframed}[style=PropertyFrame]
    \begin{prop}
    \end{prop}
    \cusTi{Inyectividad de funciones escalares}
    \cusTe{Si una función escalar es monótona, entonces esta es inyectiva:}
    \begin{gather*}
        \textrm{Dada} \hspace{1ex} f: \setR \longrightarrow \setR
        \\
        \textrm{Si} \hspace{1ex} \frac{\dif}{\dif x} f(x) \neq 0 \quad \forall x \hspace{1ex} \in \operatorname{Dn}(f)
        \\
        \Rightarrow f \hspace{1ex} \textrm{es inyectiva}
    \end{gather*}
\end{mdframed}

De manera general, para funciones de varias variables (es decir campos escalares) se estudia la inyectividad por definición: una función es inyectiva si no existen dos valores distintos del dominio que tengan una misma imagen.

\begin{mdframed}[style=DefinitionFrame]
    \begin{defn}
    \end{defn}
    \cusTi{Inyectividad}
    \begin{gather*}
        f: \setR^\nth \longrightarrow \setR \hspace{1ex} \textrm{es inyectiva si se verifica:}
        \\
        f(\Vec{x}_1)=f(\Vec{x}_2) \Rightarrow \Vec{x}_1=\Vec{x}_2
    \end{gather*}
\end{mdframed}

\begin{mdframed}[style=PropertyFrame]
    \begin{prop}
    \end{prop}
    \cusTi{Inyectividad de funciones vectoriales}
    \cusTe{Una función vectorial es inyectiva si cada una de sus funciones escalares componentes lo son.}
    \begin{multline*}
        \textrm{Dada} \hspace{1ex} \Vec{F}: \setR \longrightarrow \setR^\mth \tq
        \\
        \Vec{F}(x) = \begin{bmatrix} f_1(x) & f_2(x) & \ldots & f_\mth(x) \end{bmatrix}
    \end{multline*}
    \begin{gather*}
        \textrm{Si} \hspace{1ex} f_\ith(x) \hspace{1ex} \textrm{es inyectiva} \hspace{1ex} \forall \hspace{1ex} i \in [1;\mth]
        \\
        \Rightarrow \Vec{F} \hspace{1ex} \textrm{es inyectiva}
    \end{gather*}
\end{mdframed}

\begin{mdframed}[style=PropertyFrame]
    \begin{prop}
    \end{prop}
    \cusTi{Inyectividad de campos vectoriales}
    \cusTe{Un campo vectorial es inyectivo si cada uno de sus campos escalares componentes lo son.}
    \begin{multline*}
        \textrm{Dada} \hspace{1ex} \Vec{F}: \setR^\nth \longrightarrow \setR^\mth \tq
        \\
        \Vec{F}(\Vec{x}) = \begin{bmatrix} f_1(\Vec{x}) & f_2(\Vec{x}) & \ldots & f_\mth(\Vec{x}) \end{bmatrix}
    \end{multline*}
    \begin{gather*}
        \textrm{Si} \hspace{1ex} f_\ith(\Vec{x}) \hspace{1ex} \textrm{es inyectiva} \hspace{1ex} \forall \hspace{1ex} i \in [1;\mth]
        \\
        \Rightarrow \Vec{F} \hspace{1ex} \textrm{es inyectiva}
    \end{gather*}
\end{mdframed}

\begin{mdframed}[style=DefinitionFrame]
    \begin{defn}
    \end{defn}
    \cusTi{Sobreyectividad}
    \cusTe{Una función es sobreyectiva si el conjunto de llegada es igual al conjunto imagen.}
\end{mdframed}

\begin{mdframed}[style=PropertyFrame]
    \begin{prop}
    \end{prop}
    \cusTi{Sobreyectividad de funciones escalares}
    \begin{gather*}
        \textrm{Dada} \hspace{1ex} f: \setR \longrightarrow \setR
        \\
        \textrm{Si} \hspace{1ex} \im(f) = \setR
        \Rightarrow f \hspace{1ex} \textrm{es sobreyectiva}
        \\
        \textrm{Si} \hspace{1ex} \lim_{x \to \infty} f(x) = \pm \infty
        \Rightarrow f \hspace{1ex} \textrm{es sobreyectiva}
    \end{gather*}
\end{mdframed}

\begin{mdframed}[style=PropertyFrame]
    \begin{prop}
    \end{prop}
    \cusTi{Sobreyectividad de funciones vectoriales}
    \cusTe{Una función vectorial es sobreyectiva si cada una de sus funciones escalares componentes lo son.}
    \begin{multline*}
        \textrm{Dada} \hspace{1ex} \Vec{F}: \setR \longrightarrow \setR^\mth \tq
        \\
        \Vec{F}(x) = \begin{bmatrix} f_1(x) & f_2(x) & \ldots & f_\mth(x) \end{bmatrix}
    \end{multline*}
    \begin{gather*}
        \textrm{Si} \hspace{1ex} f_\ith(x) \hspace{1ex} \textrm{es sobreyectiva} \hspace{1ex} \forall \hspace{1ex} i \in [1;\mth]
        \\
        \Rightarrow \Vec{F} \hspace{1ex} \textrm{es sobreyectiva}
    \end{gather*}
\end{mdframed}

\begin{mdframed}[style=DefinitionFrame]
    \begin{defn}
    \end{defn}
    \cusTi{Biyectividad de campos vectoriales}
    \cusTe{Un campo vectorial es biyectivo si es inyectivo y sobreyectivo simultáneamente.
    Esto es:}
    \begin{multline*}
        \textrm{Dada} \hspace{1ex} \Vec{F}: \setR^\nth \longrightarrow \setR^\mth \tq
        \\
        \Vec{F}(\Vec{x}) = \begin{bmatrix} f_1(\Vec{x}) & f_2(\Vec{x}) & \ldots & f_\mth(\Vec{x}) \end{bmatrix} = \Vec{y}
    \end{multline*}
    \begin{gather*}
        \Vec{F} \hspace{1ex} \textrm{es biyectiva si:}
        \\
        \forall \hspace{1ex} \Vec{y} \in \setR^\mth \hspace{1ex} \exists ! \hspace{1ex} \Vec{x} \in \operatorname{Dn}(\Vec{F}): \Vec{y} = \Vec{F}(\Vec{x})
        \\[1em]
        \textrm{O bien, si:}
        \\
        \forall \hspace{1ex} \Vec{x} \in \operatorname{Dn}(\Vec{F}) \hspace{1ex} \exists ! \hspace{1ex} \Vec{y} \in \setR^m : \Vec{x} = \Vec{F}^{-1}(\Vec{y})
    \end{gather*}
\end{mdframed}


\section{Funciones homogéneas}

La homogeneidad de una función está relacionada con el grado de su fórmula.

\begin{mdframed}[style=DefinitionFrame]
    \begin{defn}
    \end{defn}
    \cusTi{Homogeneidad}
    \begin{gather*}
        f:\setR^\nth \longrightarrow \setR \hspace{1ex} \textrm{es homogénea de grado $k$ si:}
        \\
        \lambda^k \, f(\Vec{x}) = f(\lambda^k \, \Vec{x})
    \end{gather*}
\end{mdframed}

\begin{mdframed}[style=PropertyFrame]
    \begin{prop}
    \end{prop}
    Si una función es homogénea, se puede calcular el límite por reemplazo directo tras hacer un cambio de variables a coordenadas polares.
\end{mdframed}