\chapter{Integrales de funciones complejas}

\section{Curvas en el plano complejo}

Las curvas dadas en el plano complejo son funciones $(c)$ que toman valores reales $(t\in\setR)$ y los transforman en números complejos $(x + \iu y \in\setC)$.
\begin{equation*}
    c: D\subseteq \setR \longrightarrow \setC \tq c(t) = x(t) + \iu y(t)
\end{equation*}

Las curvas se derivan y se integran componente a componente.
Es decir que es válido:
\begin{gather*}
    \frac{\dif}{\dif t} c(t) = \frac{\dif}{\dif t} x(t) + \iu \frac{\dif}{\dif t} y(t)
    \\[1em]
    \int_{t_1}^{t_2} c(t) \, \dif t = \int_{t_1}^{t_2} x(t) \, \dif t + \iu \int_{t_1}^{t_2} y(t) \, \dif t
\end{gather*}

\begin{mdframed}[style=PropertyFrame]
    \begin{prop}
    \end{prop}
    \begin{equation*}
        \Re \left( \int_{t_1}^{t_2} c(t) \, \dif t \right) = \int_{t_1}^{t_2} \Re \Big( c(t) \Big) \dif t
    \end{equation*}
\end{mdframed}

Haciendo:
\begin{equation*}
    \int_{t_1}^{t_2} c(t) \, \dif t = z = \rho \, e^{\iu \theta}
\end{equation*}

\begin{align*}
    \rho &= \int_{t_1}^{t_2} e^{-\iu \theta} \, c(t) \, \dif t \quad \textrm{con} \enspace \rho = \norm{z}
    \\
    &= \Re (\rho) = \Re \left( \int_{t_1}^{t_2} e^{-\iu \theta} \, c(t) \, \dif t \right)
    \\
    &= \int_{t_1}^{t_2} \Re \left( e^{-\iu \theta} c(t) \right) \dif t 
\end{align*}

\begin{equation*}
    \rho \leq \int_{t_1}^{t_2} \norm{e^{-\iu \theta} \, c(t)} \, \dif t = \int_{t_1}^{t_2} \norm{c(t)} \, \dif t
\end{equation*}

Se obtiene la siguiente propiedad:

\begin{mdframed}[style=PropertyFrame]
    \begin{prop}
    \end{prop}
    \begin{equation*}
        \norm{\int_{t_1}^{t_2} c(t) \, \dif t} \leq \int_{t_1}^{t_2} \norm{c(t)} \, \dif t
    \end{equation*}
\end{mdframed}

\begin{mdframed}[style=DefinitionFrame]
    \begin{defn}
    \end{defn}
    \cusTi{Longitud de curva}
    \begin{equation*}
        \operatorname{long} (C) = \int_C \norm{dz} = \int_{t_1}^{t_2} \norm{\frac{\dif}{\dif t} c(t)} \, \dif t
    \end{equation*}
\end{mdframed}


\section{Integrales curvilíneas}

La integración de funciones complejas no está definida.
Las integrales de una función compleja tienen sentido si se evalúan sobre una curva del plano complejo.
La definición es parecida a las integrales de tipo 2 de cálculo real multivariable (Def. \ref{defn:type2Int}), pero en vez de tener un producto interno se tiene un producto complejo.
\begin{gather*}
    \textrm{Sea} \enspace f: \setC \longrightarrow \setC \tq f(z) = u(z) + \iu v(z)
    \\[1ex]
    \int_{C} f(z) \, \dif z = \int_{t_1}^{t_2} \left\{ f \Big( c(t) \Big) \cdot \frac{\dif}{\dif t} c(t) \right\} \dif t
    \\[1ex]
    \scale{0.98}{
    \int_{C} f(z) \, \dif z = \int_{t_1}^{t_2} \Big( u \, x' - v \, y' \Big) \dif t + \iu \int_{t_1}^{t_2} \Big( v \, x' + u \, y' \Big) \dif t
    }
\end{gather*}

Podemos reescribir la ecuación anterior definiendo dos campos vectoriales reales.
La integral curvilínea sería entonces un número complejo que tiene como partes real e imaginaria, integrales curvilíneas reales de tipo 2.1 respectivamente.

\begin{mdframed}[style=DefinitionFrame]
    \begin{defn}
        \label{defn:integral}
    \end{defn}
    \cusTi{Integral curvilínea}
    \begin{equation*}
        \int_{C} f(z) \, \dif z = \int_{C} \Vec{G} \cdot \dif \Vec{s} + \iu \int_{C} \Vec{H} \cdot \dif \Vec{s}
    \end{equation*}
    Donde:
    \begin{align*}
        \Vec{G}(x,y) &= \begin{bmatrix} u(x,y) & -v(x,y) \end{bmatrix}
        \\
        \Vec{H}(x,y) &= \begin{bmatrix} v(x,y) & u(x,y) \end{bmatrix}
    \end{align*}
\end{mdframed}

\begin{mdframed}[style=PropertyFrame]
    \begin{prop}
    \end{prop}
    \begin{equation*}
        \norm{\int_C f(z) \, \dif z} \leq M \, \operatorname{long}(C)
    \end{equation*}
\end{mdframed}


\section{Teorema de la primitiva}
\label{sec:primitive}

Una función compleja $f(z)$ tiene primitiva $F(z)$ en un subconjunto abierto $A$ del plano complejo solo si la integral es independiente de la trayectoria.
Esto implica que la integral sobre una trayectoria cerrada es nula:
\begin{equation*}
    \exists \, F(z) \iff \oint f(z) \, \dif z = 0
\end{equation*}

En tal caso, la integral se computa evaluando la primitiva en el punto final menos el inicial:
\begin{equation*}
    \int_C f(z) \, \dif z = F \big( c(t_2) \big) - F \big( c(t_1) \big)
\end{equation*}


\section{Teorema de Cauchy-Goursat}
\label{sec:CauchyGoursat}

Sea $A$ un conjunto abierto simplemente conexo y $C$ una curva cerrada y simple contenida por $A$:

\begin{center}
    \def\svgwidth{0.6\linewidth}
    \input{./images/calc-int-curv-1.pdf_tex}
\end{center}

Aplicando el Teorema de Green en cada uno de los campos vectoriales de la definición (\ref{defn:integral}) de integral curvilínea, teniendo en cuenta las ecuaciones de Cauchy-Riemann, se puede observar que los integrandos se anulan.
\begin{align*}
    \oint f(z) \, \dif z &= \oint \Vec{G} \cdot \dif \Vec{s} + \iu \oint \Vec{H} \cdot \dif \Vec{s}
    \\
    &=
    \scale{0.93}{
    \iint_C \left( -v'_x - u'_y \right) \partial x \, \partial y + \iu \iint_C \left( u'_x-v'_y \right) \partial x \, \partial y
    }
    \\
    &= 0
\end{align*}

En conclusión:

\begin{mdframed}[style=PropertyFrame]
    \begin{prop}
    \end{prop}
    Si una función compleja es derivable en $A$ entonces la integral sobre cualquier curva $C$ cerrada y simple contenida por $A$ es nula:
    \begin{equation*}
        \oint f(z) \, \dif z = 0
    \end{equation*}
\end{mdframed}

\begin{mdframed}[style=PropertyFrame]
    \begin{prop}
    \end{prop}
    Si una función verifica el teorema de Cauchy-Goursat entonces existe su primitiva, ya que al tener integral cerrada nula cumple el teorema de la primitiva.
\end{mdframed}


\section{Fórmula integral de Cauchy}
\label{sec:CauchyFormula}

Sea $A$ un conjunto abierto simplemente conexo, sea $C$ una curva cerrada y simple contenida por $A$, sea $z_0$ un punto interior de $C$:

\begin{center}
    \def\svgwidth{0.6\linewidth}
    \input{./images/calc-int-curv-2.pdf_tex}
\end{center}

\begin{mdframed}[style=PropertyFrame]
    \begin{prop}
        \label{prop:CauchyFormula}
    \end{prop}
    Si $f(z)$ es derivable en todo el interior de $A$, salvo en $z_0$, se tiene que:
    \begin{equation*}
        \oint \frac{f(z)}{z - z_0} \dif z = f(z_0) \, 2 \pi \iu
    \end{equation*}
\end{mdframed}

Nótese que si $z_0$ estuviese afuera de $C$ o si $f(z)$ fuese derivable en $z_0$, la integral sobre $C$ se anularía por el Teorema de Cauchy-Goursat (Sec. \ref{sec:CauchyGoursat}).

La propiedad anterior puede ser generalizada.
\begin{equation*}
    \textrm{Sea} \enspace g(z) = \oint \frac{f(w)}{w - z} \dif w
\end{equation*}

Si $f(w)$ es contínua sobre la curva, se tiene que:
\begin{align*}
    \frac{\dif}{\dif z} g(z) &= \frac{\dif}{\dif z} \oint \frac{f(w)}{w-z}
    \\[1ex]
    &= \oint \left[ \frac{\dif}{\dif z} \frac{f(w)}{w-z} \right] \dif w
    \\[1ex]
    &= \oint \dfrac{f(w)}{(w-z)^2} dw
\end{align*}

Por extrapolación, se tiene:
\begin{equation*}
    \frac{\dif^\nth}{\dif z^\nth} g(z) = \nth! \oint \frac{f(w)}{(w-z)^{\nth+1}} \dif w
\end{equation*}

Por lo tanto $g(z) \in \class[\infty]$ y reemplazando $g(z)$ en la fórmula de Cauchy (Prop. \ref{prop:CauchyFormula}) se tiene:
\begin{equation*}
    g(z) = 2 \pi \, f(z)
\end{equation*}

Finalmente, derivando la ecuación anterior $\nth$ veces, se tiene la generalización del teorema.
Además, esto implica que $f(z) \in \mathcal{C} ^ \infty \big( \operatorname{int}(C) \big)$.
Nótese que si $\nth=0$ queda la Fórmula de Cauchy dada en la propiedad \ref{prop:CauchyFormula}.

\begin{mdframed}[style=PropertyFrame]
    \begin{prop}
    \end{prop}
    \begin{equation*}
        \frac{\dif^\nth}{\dif z^\nth} f(z_0) = \frac{\nth!}{2 \pi \iu} \oint \frac{f(z)}{(z-z_0)^{\nth+1}} \dif z
    \end{equation*}
\end{mdframed}