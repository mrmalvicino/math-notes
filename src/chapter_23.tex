\chapter{Polos}

\section{Singularidades aisladas}

Una singularidad es un punto donde una función o bien no está definida o bien no es derivable.
Si $f(z)$ tiene una singularidad en $z_0$ pero es derivable en un entorno $0<\norm{z-z_0}<R$ entonces se dice que es una singularidad aislada y $f(z)$ admite desarrollo en Series de Laurent (Sec. \ref{sec:Laurent}).
Analizando los coeficientes $b_\nth$ del desarrollo de Laurent, se pueden clasificar las singularidades aisladas de la siguiente manera:

\begin{itemize}
    \item Singularidad evitable:
    \begin{equation*}
        b_\nth = 0 \enspace \forall \enspace n>0
    \end{equation*}
    \item Polo de orden $\kth$:
    \begin{equation*}
        b_\nth = 0 \enspace \forall \enspace n>k
    \end{equation*}
    \item Singularidad esencial:
    \begin{equation*}
        b_\nth \neq 0 \enspace \forall \enspace n>0
    \end{equation*}
\end{itemize}


\section{Clasificación de polos}

A veces es posible clasificar la singularidad aislada sin hacer el desarrollo en serie de la función.
En particular, se puede detectar si una singularidad aislada es un polo de la siguiente forma.

Si $z_0$ es un Polo de orden $\kth$, la función se va a desarrollar de la siguiente manera:
\begin{multline*}
    f(z) = \frac{b_\kth}{\left(z-z_0\right)^\kth} + \frac{b_{\kth-1}}{\left(z-z_0\right)^{\kth-1}} + \frac{b_{\kth-2}}{\left(z-z_0\right)^{\kth-2}} +
    \\
    + \dots + \frac{b_1}{\left(z-z_0\right)} + \sum_{\nth=0}^\infty a_\nth \left(z-z_0\right)^\nth
\end{multline*}

\begin{multline*}
    f(z) = \frac{1}{\left(z-z_0\right)^\kth} \Big( b_\kth + b_{\kth-1} \left(z-z_0\right) + b_{\kth-2} \left(z-z_0\right)^2 +
    \\
    + \dots + b_1 \left(z-z_0\right)^{\kth-1} + \sum_{n=0}^\infty a_\nth \left(z-z_0\right)^{\nth+\kth} \Big)
\end{multline*}

Si llamamos $\Phi(z)$ a todo lo que está entre paréntesis, se tiene:
\begin{equation*}
    f(z) = \frac{\Phi(z)}{\left(z-z_0\right)^\kth}
\end{equation*}

Donde:
\begin{equation*}
    \Phi (z_0) = b_\kth \neq 0 \quad \land \quad \Phi (z_0) \in \class
\end{equation*}


\section{Ceros de una función analítica}

Se dice que $z_0$ es un cero de una $f(z)$ si verifica:
\begin{equation*}
    f(z_0)=0
\end{equation*}

Si la función es analítica en $z_0$ y su entorno, tiene desarrollo de Taylor.
\begin{equation*}
    f(z) = \sum_{\nth=0}^\infty a_\nth \left(z-z_0\right)^\nth
\end{equation*}

Cuando $z=z_0$, la función toma el primer término $a_0$ de la serie.
Por lo tanto, dado que $f(z_0) = a_0 = 0$, el desarrollo de Taylor comienza en 1.
\begin{equation*}
    f(z) = \underbrace{a_0}_{f(z_0)=0} + \sum_{\nth=1}^\infty a_\nth (z-z_0)^\nth
\end{equation*}

Pero además, la función puede tener derivadas sucesivas nulas.
Recordar que en un desarrollo de Taylor, los coeficientes $a_\nth$ están dados por las derivadas sucesivas evaluadas en $z_0$.
Si una derivada es nula, entonces también lo será este coeficiente.
\begin{align*}
    \frac{\dif}{\dif z} f(z_0) &= \frac{\dif^2}{\dif z^2} f(z_0)
    \\[1ex]
    &= \frac{\dif^{\kth-1}}{\dif z^{\kth-1}} f(z_0)
    \\
    &= 0
\end{align*}

Siguiendo el razonamiento anterior, si hay $\kth$ derivadas que se anulan entonces el desarrollo comienza en $\kth$, ya que los términos anteriores son nulos.
Tiene que haber al menos un valor de $\kth$ para el que la derivada no se anule, sino la función sería nula.
\begin{equation*}
    f(z) = \sum_{\nth=\kth}^\infty a_\nth \left(z-z_0\right)^\nth
\end{equation*}

Sacando factor común $\left(z-z_0\right)^\kth$ y definiendo la serie de potencias positivas como $\Phi(z)$, se tiene:
\begin{equation*}
    f(z) = \left(z-z_0\right)^\kth \underbrace{\sum_{\nth=\kth}^\infty a_\nth \left(z-z_0\right)^{\nth-\kth}}_{\Phi (z)}
\end{equation*}

\begin{mdframed}[style=PropertyFrame]
    \begin{prop}
    \end{prop}
    Si $z_0$ es un cero de orden $k$, entonces $f$ se puede expresar como:
    \begin{equation*}
        f(z) = \left(z-z_0\right)^\kth \Phi (z)
    \end{equation*}
    \noTi{Donde:}
    \begin{equation*}
        \Phi (z_0) = a_\kth \neq 0
    \end{equation*}
\end{mdframed}


\section{Clasificación de singularidades}

De manera general, analizando los ceros de un cociente se pueden clasificar todas las singularidades, incluyendo los polos.

Si $f(z) = h(z)/g(z)$ con $g(z_0)=0$ entonces $f(z)$ tiene una singularidad aislada en $z_0$.
Y como $g(z)$ es analítica, se puede expresar como $g(z)=\left(z-z_0\right)^\kth \Phi_g (z)$.

\begin{itemize}
    \item \textbf{Caso 1}
    
    Si $h(z_0) \neq 0$ entonces $z_0$ es un polo de orden $k$.
    \begin{align*}
        f(z) &= \frac{h(z)}{g(z)}
        \\[1ex]
        &= \underbrace{\frac{h(z)}{\Phi_g (z)}}_{\Phi_f (z)} \frac{1}{\left(z-z_0\right)^\kth}
        \\[1ex]
        &= \frac{\Phi_f (z)}{\left(z-z_0\right)^\kth}
    \end{align*}

    \item \textbf{Caso 2}
    
    Si $h(z_0) = 0$ se tiene:
    \begin{align*}
        f(z) &= \frac{\left(z-z_0\right)^\mth}{\left(z-z_0\right)^\kth} \underbrace{\frac{\Phi_h (z)}{\Phi_g (z)}}_{\Phi_f (z)}
        \\[1ex]
        &= \left(z-z_0\right)^{\mth-\kth} \Phi_f (z)
    \end{align*}
    
    \begin{itemize}
        \item \textbf{Caso 2.1}
        
        Si $m \geq k$ entonces $z_0$ es una singularidad aislada evitable.
        \begin{equation*}
            f(z) = \left(z-z_0\right)^{\mth-\kth} \Phi(z)
        \end{equation*}

        \item \textbf{Caso 2.2}
        
        Si $m<k$ entonces $z_0$ es un polo de orden $k-m$.
        \begin{equation*}
            f(z) = \frac{\Phi (z)}{\left(z-z_0\right)^{\kth-\mth}}
        \end{equation*}
    \end{itemize}
\end{itemize}