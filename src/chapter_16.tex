\chapter{Teoremas del cálculo integral}

\section{Teo. de Green-Riemann}

El Teorema de Green relaciona la integral curvilínea de tipo 2.1 con la integral doble de la tercera componente del rotacional.

Sea $C$ una curva cerrada y simple, orientada positivamente y sea $A$ un recinto simplemente conexo, se tiene que:
\begin{align*}
    \oint_C \Vec{F} \cdot \dif \Vec{s}
    &= \iint_A \left( \grad \times \Vec{F} \right) \kVer \, \partial x \, \partial y
    \\[1ex]
    &= \left( \dfrac{\partial}{\partial x} F_2 - \dfrac{\partial}{\partial y} F_1 \right) \partial x \, \partial y
\end{align*}

\section{Teo. de Stokes}

El Teorema de Stokes relaciona la integral curvilínea de tipo 2.1 con la integral de superficie del rotacional.

Sea $\Vec{F}$ un campo vectorial de clase $\class (\setR^3)$ y sea $S$ una superficie cerrada, simple, diferenciable y compatible con la orientación, se tiene que:
\begin{equation*}
    \oint_C \Vec{F} \cdot \dif \Vec{s} = \oiint_S \left( \grad \times \Vec{F} \right) \dif \Vec{S}
\end{equation*}

\section{Teo. de Gauss-Ostrogradsky}

El Teorema de Gauss relaciona la integral de superficie de tipo 2.2 con la divergencia de una fuerza en un volumen.

Sea $\Vec{F}$ un campo vectorial de clase $\class (\setR^3)$ sea $S$ una superficie cerrada, simple, diferenciable y compatible con la orientación, se tiene que:
\begin{equation*}
    \oiint_S \Vec{F} \cdot \dif \Vec{S} = \iiint_V \left( \grad \cdot \Vec{F} \right) \partial x \, \partial y \, \partial z
\end{equation*}