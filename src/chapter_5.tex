\chapter{Transformaciones lineales}
\label{cha:TL}

\begin{mdframed}[style=DefinitionFrame]
    \begin{defn}
        \label{defn:TL}
    \end{defn}
    \cusTi{Transformación lineal}
    \cusTe{Sean $\setV$ y $\setW$ dos espacios vectoriales, se dice que una función $f:\setV\longrightarrow\setW$ es una transformación lineal si para todo $\Vec{v}_1,\Vec{v}_2\in\setV$ y $\lambda\in\setR$ se verifica:}
    \begin{equation*}
        f(\lambda \, \Vec{v}_1 + \Vec{v}_2) = \lambda \, f(\Vec{v}_1) + f(\Vec{v}_2)
    \end{equation*}
\end{mdframed}

Observar que si la definición se cumple para todo $\lambda$ y para todo $\Vec{v}$ entonces se cumple particularmente para $\lambda=1$ y para $\Vec{v}_2=\Vec{0}$.
Es decir que si se cumple $f(\lambda \, \Vec{v}_1 + \Vec{v}_2)=\lambda \, f(\Vec{v}_1) + f(\Vec{v}_2)$ entonces se cumplen las siguientes condiciones simultáneamente:
\begin{equation*}
    \left\{
    \begin{aligned}
        f(\Vec{v}_1 + \Vec{v}_2) &= f(\Vec{v}_1) + f(\Vec{v}_2)
        \\
        f(\lambda \, \Vec{v}) &= \lambda \, f(\Vec{v})
    \end{aligned}
    \right.
\end{equation*}

\begin{mdframed}[style=PropertyFrame]
    \begin{prop}
    \end{prop}
    \cusTi{Condición necesaria}
    \cusTe{Si $f:\setV\longrightarrow\setW$ es una transformación lineal entonces:}
    \begin{equation*}
        f(\Vec{0}_\setV) = \Vec{0}_\setW
    \end{equation*}
\end{mdframed}

Una transformación lineal puede ser expresada por su fórmula:
\begin{equation*}
    f: \setV \longrightarrow \setW \tq f(\Vec{v}) = \Vec{w}
\end{equation*}

Puede ser expresada mediante \emph{la matriz de la transformación} 
\begin{equation*}
    f: \setV \longrightarrow \setW \tq M_{EE}(f) \cdot \Vec{v} = \Vec{w}
\end{equation*}

Pero además, puede ser expresada por los transformados de una base.

\begin{mdframed}[style=PropertyFrame]
    \begin{prop}
    \end{prop}
    \cusTi{Teorema fundamental de existencia y unicidad de las transformaciones lineales}
    \cusTe{Sean $\setV$ y $\setW$ dos espacios vectoriales, $B=\inBraces{\Vec{v}_1 , \Vec{v}_2 \dots \Vec{v}_\nth}$ una base de $\setV$ y $\inBraces{\Vec{w}_1 , \Vec{w}_2 \dots \Vec{w}_\nth} \in \setW$ entonces $\exists ! \enspace f: \setV \longrightarrow \setW$ que es transformación lineal tal que:}
    \begin{gather*}
        f(\Vec{v}_1) = \Vec{w}_1
        \\
        f(\Vec{v}_2) = \Vec{w}_2
        \\
        \vdots
        \\
        f(\Vec{v}_\nth) = \Vec{w}_\nth
    \end{gather*}
\end{mdframed}

\concept{Demostración}

Para cierto $\Vec{v} \in \setV$, como $B$ es base de $\setV$, se tiene:
\begin{equation*}
    \exists ! \enspace \lambda_\ith \hspace{1ex} \textrm{con} \hspace{1ex} 1 \leq \ith \leq \nth \tq
    \Vec{v} = \lambda_1 \, \Vec{v}_1 + \lambda_2 \, \Vec{v}_2 + \dots + \lambda_\nth \, \Vec{v}_\nth
\end{equation*}

Entonces:
\begin{align*}
    f(\Vec{v}) &= f \left( \lambda_1 \, \Vec{v}_1 + \lambda_2 \, \Vec{v}_2 + \dots + \lambda_\nth \, \Vec{v}_\nth \right)
    \\
    &= \lambda_1 \, f(\Vec{v}_1) + \lambda_2 \, f(\Vec{v}_2) + \dots + \lambda_\nth \, f(\Vec{v}_\nth)
    \\
    &= \lambda_1 \, \Vec{w}_1 + \lambda_2 \, \Vec{w}_2 + \dots + \lambda_\nth \, \Vec{w}_\nth
    \\
    &=
    \begin{pmatrix}
        \trans{\Vec{w}_1} & \trans{\Vec{w}_2} & \dots & \trans{\Vec{w}_\nth}
    \end{pmatrix}
    \cdot
    \begin{pmatrix}
        \lambda_1
        \\
        \lambda_2
        \\
        \vdots
        \\
        \lambda_\nth
    \end{pmatrix}
    \\[1ex]
    &=
    \begin{pmatrix}
        w_{11} & w_{21} & \dots & w_{\nth 1}
        \\
        w_{12} & w_{22} & \dots & w_{\nth 2}
        \\
        \vdots & \vdots & \ddots & \vdots
        \\
        w_{1 \mth} & w_{2 \mth} & \dots & w_{\nth \mth}
    \end{pmatrix}
    \cdot
    \begin{pmatrix}
        \lambda_1
        \\
        \lambda_2
        \\
        \vdots
        \\
        \lambda_\nth
    \end{pmatrix}
    \\[1ex]
    &= M_{BE} (f) \cdot \trans{[\Vec{v}]_B}
\end{align*}

Donde $\nth=\operatorname{dim}(\setV)$ y $\mth=\operatorname{dim}(\setW)$.

La matriz $M_{BE}$ de la ecuación anterior tiene como columnas a las imágenes de los elementos de la base $B$.
Si esta base fuese la canónica, la matriz obtenida sería $M_{EE}(f)$ y estaría dada por las imágenes de los versores canónicos.

Pero en caso de no tener $M(f)$, sino $M_{BE}(f)$, el inconveniente que surge es que las variables que querramos evaluar en $f(\Vec{v})$ van a estar en base $E$ pero las variables de la fórmula en forma matricial en base $B$.
En tal caso, podemos usar las matrices de cambio de base (Def. \ref{defn:C_BE}) para obtener las coordenadas de $\Vec{v}$ en base canónica.
\begin{align*}
    f(\Vec{v}) &= M_{BE} (f) \cdot \trans{[\Vec{v}]_B}
    \\
    &= M_{BE} (f) \cdot C_{EB} \cdot \trans{[\Vec{v}]_E}
    \\
    &= M_{BE} (f) \cdot C_{EB} \cdot \trans{\Vec{v}}
    \\
    &= M(f) \cdot \trans{\Vec{v}}
\end{align*}

Donde:
\begin{align*}
    C_{EB} &=
    \begin{pmatrix}
        \trans{\Vec{v}_1} & \trans{\Vec{v}_2} & \dots & \trans{\Vec{v}_\nth}
    \end{pmatrix}^{-1}
    \\
    &=
    \begin{pmatrix}
        v_{11} & v_{21} & \dots & v_{\nth 1}
        \\
        v_{12} & v_{22} & \dots & v_{\nth 2}
        \\
        \vdots & \vdots & \ddots & \vdots
        \\
        v_{1 \nth} & v_{2 \nth} & \dots & v_{\nth \nth}
    \end{pmatrix}^{-1}
\end{align*}

Observar que los $\Vec{w}$ anteriores están en base canónica, pero podrían estarlo en otra base.
Podemos definir una transformación matricialmente en bases $A$ y $B$ genéricas de la siguiente forma.

Notar que en tal caso $A$ cumple el papel que antes cumplía $B$, ya que $B$ cumplirá el papel que antes cumplía $E$.

\begin{mdframed}[style=DefinitionFrame]
    \begin{defn}
    \end{defn}
    \cusTi{Forma matricial de una TL en bases $AB$}
    \cusTe{Sean $A=\inBraces{\Vec{v}_1 , \Vec{v}_2 \dots \Vec{v}_\nth}$ una base de $\setV$ y $B=\inBraces{\Vec{w}_1 , \Vec{w}_2 \dots \Vec{w}_\mth}$ una base de $\setW$ se define la transformación lineal}
    \begin{equation*}
        f: \setV \longrightarrow \setW \tq M_{AB}(f) \cdot \inBrackets{\Vec{v}}_A = \inBrackets{\Vec{w}}_B
    \end{equation*}
    \noTi{Donde:}
    \begin{equation*}
        M_{AB} (f) =
        \begin{pmatrix}
            \trans{\left[f(\Vec{v}_1)\right]_{B}} &
            \trans{\left[f(\Vec{v}_2)\right]_{B}} &
            \dots &
            \trans{\left[f(\Vec{v}_\nth)\right]_{B}}
        \end{pmatrix}
    \end{equation*}
\end{mdframed}

Observar que la matriz $M_{AB}$ tiene $\nth$ columnas según la dimensión de $\setV$ y $\mth$ filas según la dimensión de $\setW$.

Podemos usar las matrices de cambio de base (Def. \ref{defn:C_BB'}) para expresar $M_{AB}$ en bases $A'$ y $B'$.
\begin{align*}
    M_{A'B'}(f) \cdot \inBrackets{\Vec{v}}_{A'} &= \inBrackets{\Vec{w}}_{B'}
    \\
    C_{BB'} \cdot M_{AB}(f) \cdot C_{A'A} \cdot \inBrackets{\Vec{v}}_{A'} &= \inBrackets{\Vec{w}}_{B'}
\end{align*}

De manera que:

\begin{mdframed}[style=PropertyFrame]
    \begin{prop}
    \end{prop}
    \begin{equation*}
        M_{A'B'}(f) = C_{BB'} \cdot M_{AB}(f) \cdot C_{A'A}
    \end{equation*}
\end{mdframed}

O análogamente:
\begin{align*}
    M_{AB}(f) \cdot \inBrackets{\Vec{v}}_A &= \inBrackets{\Vec{w}}_B
    \\
    C_{B'B} \cdot M_{A'B'}(f) \cdot C_{AA'} \cdot \inBrackets{\Vec{v}}_A &= \inBrackets{\Vec{w}}_B
\end{align*}

Obteniendo:

\begin{mdframed}[style=PropertyFrame]
    \begin{prop}
        \label{prop:M_BB}
    \end{prop}
    \begin{equation*}
        M_{AB}(f) = C_{B'B} \cdot M_{A'B'}(f) \cdot C_{AA'}
    \end{equation*}
\end{mdframed}


\section{Imagen y núcleo}

\begin{mdframed}[style=DefinitionFrame]
    \begin{defn}
    \end{defn}
    \cusTi{Imagen}
    \cusTe{Es el conjunto de vectores que resultan de aplicar la transformación a todos los elementos del dominio.}
    \begin{equation*}
        \im(f) = \inBraces{\Vec{w} \in \setW \tq \exists \, \Vec{v} \in \setV \enspace \textrm{con} \enspace f(\Vec{v}) = \Vec{w}}
    \end{equation*}
\end{mdframed}

Para calcular la imagen se aplica la transformación a cada elemento de una base $B=\inBraces{\Vec{v}_1,\Vec{v}_2 \dots \Vec{v}_\nth}$ de $\setV$ y así otener un conjunto generador de la imagen.
Observar que para cualquier elemento $\Vec{v}\in\setV$ se verifica:
\begin{gather*}
    f(\Vec{v}) = f(\lambda_1 \, \Vec{v}_1 + \lambda_2 \, \Vec{v}_2 + \dots + \lambda_\nth \, \Vec{v}_\nth)
    \\
    f(\Vec{v}) = \lambda_1 \, f(\Vec{v}_1) + \lambda_2 \, f(\Vec{v}_2) + \dots + \lambda_\nth \, f(\Vec{v}_\nth)
    \\
    \im(f) = \operatorname{gen}\inBraces{f(\Vec{v}_1),f(\Vec{v}_2) \dots f(\Vec{v}_\nth)}
\end{gather*}

\begin{mdframed}[style=ExampleFrame]
    \begin{example}
    \end{example}
    Sea $f:\operatorname{dn}(f) \subseteq \setR^3 \longrightarrow \im(f) \subseteq \setR^2$ una T.L. tal que $f(x,y,z)=(x+y,y-z)$, se pide calcular la imagen.
    
    Aplicando la transformación a los elementos de la base canónica se tiene:
    \begin{align*}
        f(\eVer_1) &= (1,0)
        \\
        f(\eVer_2) &= (1,1)
        \\
        f(\eVer_3) &= (0,-1)
    \end{align*}
    
    Por lo tanto el conjunto generador de la imagen de $f$ es:
    \begin{equation*}
        \im(f) = \operatorname{gen}\inBraces{(1,0),(1,1),(0,-1)}
    \end{equation*}
    
    O bien, descartando $(0,-1)=(1,0)-(1,1)$ por ser combinación lineal se tiene:
    \begin{equation*}
        \im(f) = \operatorname{gen}\inBraces{(1,0),(1,1)} = \setR^2
    \end{equation*}
\end{mdframed}

\begin{mdframed}[style=DefinitionFrame]
    \begin{defn}
    \end{defn}
    \cusTi{Núcleo}
    \cusTe{Es el conjunto de vectores que tienen como imagen el origen.}
    \begin{equation*}
        \Nu(f) = \inBraces{\Vec{v} \in \setV \tq f(\Vec{v})=\Vec{0}}
    \end{equation*}
\end{mdframed}

\begin{mdframed}[style=PropertyFrame]
    \begin{prop}
    \end{prop}
    \cusTi{Teorema de la dimensión}
    \cusTe{Sea $f:\setV\longrightarrow\setW$ una transformación lineal tal que la dimensión de $\setV$ es finita entonces:}
    \begin{equation*}
        \operatorname{dim}(\setV) = \operatorname{dim} \big(\Nu(f)\big) + \operatorname{dim} \big(\im(f)\big)
    \end{equation*}
\end{mdframed}


\section{Clasificación de T.L.}

Si $f: \setV \longrightarrow \setW$ es una T.L. basta con que el núcleo esté compuesto solamente por el vector origen para que $f$ sea inyectiva.
Esto implica que no existe T.L. cuyo núcleo sea el origen, y además tenga una misma imagen para dos elementos de $\setV$ distintos.
Esto es:
\begin{gather*}
    \textrm{Sea} \enspace f: \setV \longrightarrow \setW \enspace \textrm{una T.L.}
    \\
    \textrm{Sean} \enspace \Vec{v}_1,\Vec{v}_2 \in \setV \tq \Vec{v}_1 \neq \Vec{v}_2 \land f(\Vec{v}_1) = f(\Vec{v}_2) \neq \Vec{0}
    \\
    \textrm{Observar que} \enspace \Vec{v}_1,\Vec{v}_2 \notin \Nu(f)
    \\
    f(\Vec{v}_1) - f(\Vec{v}_2) = \Vec{0}
    \\
    f(\Vec{v}_1 - \Vec{v}_2) = \Vec{0}
    \\
    \therefore \left( \Vec{v}_1 - \Vec{v}_2 \right) \in \Nu(f) \neq \Vec{0}
\end{gather*}


\begin{mdframed}[style=DefinitionFrame]
    \begin{defn}
    \end{defn}
    \cusTi{Monomorfismo}
    \begin{multline*}
        f \enspace \textrm{es un monomorfismo} \iff
        \\
        \iff f  \enspace \textrm{es inyectiva}
        \iff \Nu(f) - \Vec{0} = \setO
    \end{multline*}
\end{mdframed}

\begin{mdframed}[style=DefinitionFrame]
    \begin{defn}
    \end{defn}
    \cusTi{Epimorfismo}
    \begin{multline*}
        f \enspace \textrm{es un epimorfismo} \iff
        \\
        \iff f  \enspace \textrm{es sobreyectiva}
        \iff \im(f) = \setW
    \end{multline*}
\end{mdframed}

\begin{mdframed}[style=DefinitionFrame]
    \begin{defn}
    \end{defn}
    \cusTi{Isomorfismo}
    \begin{equation*}
        f \enspace \textrm{es un isomorfismo} \iff f \enspace \textrm{es biyectiva}
    \end{equation*}
\end{mdframed}

\begin{mdframed}[style=DefinitionFrame]
    \begin{defn}
    \end{defn}
    \cusTi{Endomorfismo}
    \begin{equation*}
        f \enspace \textrm{es un endomorfismo} \iff \setV = \setW
    \end{equation*}
\end{mdframed}

\begin{mdframed}[style=DefinitionFrame]
    \begin{defn}
    \end{defn}
    \cusTi{Automorfismo}
    \begin{multline*}
        f \enspace \textrm{es un automorfismo} \iff
        \\
        \iff f \enspace \textrm{es isomorfismo y endomorfismo}
    \end{multline*}
\end{mdframed}