\chapter{Integrales de funciones reales}

Al igual que la derivación, la integración solo puede operar con una variable a la vez.
Pero a diferencia de las derivadas, es posible computar una unica integral y llegar a un resultado numérico integrando con respecto de todas las variables.

Según cómo esté definida la integración va a variar conceptualmente su aplicación.
Integrar funciones de una variable equivale a calcular el área debajo de la curva.
Integrar funciones de varias variables también es posible, y en el caso de un campo escalar de dos variables, la integral calcula el volumen por debajo de la superficie.
Aunque esta noción de la integral es una primera idea intuitiva, ya que la integración tiene muchas otras aplicaciones.

\section{Integrales de Riemann en 2D}

A continuación, se ve que una integral sobre un recinto rectangular $R$ calcula el volumen que hay por debajo del gráfico de un campo escalar de dos variables.

Sea $f$ una función contínua y positiva que no vale constantemente uno:
\begin{equation*}
    f:D \subseteq \setR^2 \longrightarrow \setR \tq 1 \neq f(x,y) > 0
\end{equation*}

Sea $R$, una partición de $D$, el área dada por el producto cartesiano entre intervalos de los ejes $x$ e $y$:
\begin{equation*}
    R = [a,b] \times [c,d] = [x_0,x_n] \times [y_0,y_n]
\end{equation*}

Nótese que el recinto $R$ es un rectángulo de puntos $\begin{bmatrix} x & y \end{bmatrix} \in \setR^2$ que pertenecen a un plano en el dominio de la función.
Mientras que los puntos de la superficie $S$ están definidos como $\begin{bmatrix} x & y & f(x,y) \end{bmatrix} \in \setR^3$ y pertenecen al espacio.

Esto implica que los puntos de la superficie $S$ dada por el gráfico de $f(x,y)$ no solamente están ``elevados'' con respecto a los puntos del recinto $R$.
Si no que, además, no tienen porqué tener la misma geometría.
De hecho, ni siquera pertenecen al mismo espacio vectorial.

Justamente el hecho de que tengan otro ``relieve'' va a permitir definir un volumen que no sea el de un cubo, donde bastaría con multiplicar sus lados para calcular el volumen.
Por este motivo, se definió $f(x,y) \neq 1$.

Una vez que se tiene el recinto particionado, se establece para cara partición un paralelepípedo de base igual al área de la partición y de altura tal que uno de sus vértices ``toque'' la gráfica de la función.

En la siguiente imagen se muestran graficados la superficie $S$ que es la gráfica de $f(x,y)$, un recinto $R$ de $n \times n = 4 \times 4 = 16$ particiones pudiendo $n$ ser cualquier número natural, y uno de los 16 paralelepípedos:

\begin{center}
    \def\svgwidth{0.6\linewidth}
    \input{./images/calc-integral-1.pdf_tex}
\end{center}

El volumen del paralelepípedo graficado es:
\begin{equation*}
    V_{3,0}= \underbrace{(x_4-x_3)}_{\textrm{largo}} \times \underbrace{(y_1-y_0)}_{\textrm{ancho}} \times \underbrace{f(x_3,y_0)}_{\textrm{alto}}
\end{equation*}

El volumen de cualquier paralelepípedo es:
\begin{equation*}
    V_{\ith,\jth}= \underbrace{(x_{\ith+1} - x_\ith)}_{\Delta x_\ith} \times \underbrace{(y_{\jth+1} - y_\jth)}_{\Delta y_\jth} \times f(x_\ith,y_\jth)
\end{equation*}

El volumen de todos los paralelepípedos del recinto es la suma de cada volumen.
Esta aproximación del volumen total $V$ que hay por debajo de $S$ es:
\begin{equation*}
    V \approx \sum_{\ith,\jth=0}^{\nth-1} V_{\ith,\jth} = \sum_{\ith,\jth=0}^{\nth-1} f(x_\ith,y_\jth) \, \Delta x_\ith \, \Delta y_\jth
\end{equation*}

El largo $\Delta x_\ith$ y el ancho $\Delta y_\jth$ de cada paralelepípedo depende de la cantidad $\nth$ de particiones del recinto y de los extremos $a$, $b$, $c$ y $d$ de los intervalos:
\begin{equation*}
    \left\{
    \begin{aligned}
        \Delta x_\ith &=& x_{\ith+1} - x_\ith = \dfrac{b-a}{\nth}
        \\[1em]
        \Delta y_\jth &=& y_{\jth+1} - y_\jth = \dfrac{d-c}{\nth}
    \end{aligned}
    \right.
\end{equation*}

Al aumentar $\nth$ aumenta la cantidad de particiones y por ende la cantidad de paralelepípedos, no así el tamaño del recinto.
Por lo tanto, el grosor de los paralelepipedos tiene que disminuir.
Tomar el límite cuando $n\to\infty$ implica sumar los volúmenes de los infinitos paralelepípedos de grosor nulo que hay en el recinto.
De esta forma, la aproximación del volumen bajo la superficie pasa a ser una igualdad, quedando definida una integral doble:
\begin{align*}
    V &= \lim_{\nth\to\infty} \sum_{\ith,\jth=0}^{\nth-1} f(x_\ith,y_\jth) \, \Delta x_\ith \, \Delta y_\jth
    \\
    &= \int_c^d \int_a^b f(x,y) \, \partial x \, \partial y
\end{align*}

\begin{mdframed}[style=PropertyFrame]
    \begin{prop}
    \end{prop}
    \cusTi{Teorema de Fubini}
    \cusTe{Si el recinto de integración es rectangular, entonces los extremos de los intervalos son permutables.}
\end{mdframed}


\section*{Cambio de variables en 2D}

Sea $\Vec{T}:A' \subset \setR^2 \longrightarrow A \subset \setR^2$ una transformación de clase $\Vec{T} \in \class (A')$ y sea el determinante de la matriz jacobiana $\operatorname{det}(J \, \Vec{T}(u,v)) \neq 0$, se tiene que:
\begin{multline*}
    \iint_A f(x,y) \, \partial x \, \partial y =
    \\
    = \iint_{A'} f \left( \Vec{T}(u,v) \right) \norm{ det(J \Vec{T}(u,v)) } \, \partial u \, \partial v
\end{multline*}

\begin{center}
    \def\svgwidth{0.8\linewidth}
    \input{./images/calc-transformacion.pdf_tex}
\end{center}

% \section*{Transformaciones lineales}
% \section*{Coordenadas polares}
% \section*{Coordenadas elípticas}
% \section*{Cambio de variables en 3D}
% \section*{Coordenadas cilíndricas}
% \section*{Coordenadas esféricas}


\section{Aplicaciones}


\subsection{Distancia lineal}

Dada la función $f:D \subset \setR \longrightarrow \setR \tq f(x)=1$.
El cálculo de la integral sobre un intervalo $\Delta x$ da como resultado el largo del intervalo.
\begin{equation*}
    A = \int_{x_0}^{x_1} \dif x = 1 \cdot \Delta x
\end{equation*}

\begin{center}
    \def\svgwidth{0.6\linewidth}
    \input{./images/calc-integral-2.pdf_tex}
\end{center}


\subsection{Área bajo una curva plana}

Dada una función $f:D \subset \setR \longrightarrow \setR \tq f(x)>0$.
El cálculo de la integral sobre un intervalo $\Delta x$ da como resultado el área bajo la gráfica de $f$.
\begin{equation*}
    A = \int_{x_0}^{x_1} f(x) \, \dif x
\end{equation*}

\begin{center}
    \def\svgwidth{0.6\linewidth}
    \input{./images/calc-integral-3.pdf_tex}
\end{center}


\subsection{Área del recinto $A$}

Dada la función $f:D \subset \setR^2 \longrightarrow \setR \tq f(x,y)=1$.
El cálculo de la integral sobre un recinto $A$ da como resultado el área del recinto.
\begin{equation*}
    A = \iint_A \partial \, x \partial y = 1 \cdot \Delta x \, \Delta y
\end{equation*}

\begin{center}
    \def\svgwidth{0.6\linewidth}
    \input{./images/calc-integral-4.pdf_tex}
\end{center}


\subsection{Volumen bajo una superficie}

Dada una función $f:D \subset \setR^2 \longrightarrow \setR \tq f(x,y)>0$.
El cálculo de la integral sobre un recinto $A$ da como resultado el volumen bajo la gráfica de $f$.
\begin{equation*}
    V = \iint_A f(x,y) \partial x \, \partial y
\end{equation*}

\begin{center}
    \def\svgwidth{0.6\linewidth}
    \input{./images/calc-integral-5.pdf_tex}
\end{center}


\subsection{Volumen del recinto $V$}

Dada la función $f:D \subset \setR^3 \longrightarrow \setR \tq f(x,y,z)=1$.
El cálculo de la integral sobre un volumen de integración $V$ da como resultado dicho volumen.
\begin{equation*}
    V = \iiint_V \partial x \, \partial y \, \partial z = 1 \cdot \Delta x \, \Delta y \, \Delta z
\end{equation*}


\subsection{Longitud de una curva plana}

El largo $L$ de un arco de curva plana está dado por:
\begin{equation*}
    L = \int_{t_0}^{t_1} \nnorm{ \dfrac{\dif}{\dif t} \Vec{c}(t) \dif t } \, \dif t
\end{equation*}


\subsection{Área de una superficie}

El área $A$ de una superficie está dada por:
\begin{align*}
    A &= \iint_S \nnorm{\Vec{n}(u,v)} \, \partial u \, \partial v
    \\[1ex]
    &= \iint_S \nnorm{ \dfrac{\partial}{\partial u} \Vec{s}(u,v) \times \dfrac{\partial}{\partial v} \Vec{s}(u,v) } \, \partial u \, \partial v
\end{align*}


\subsection{Masa de un sólido}

Sea $\rho:\setR^3 \longrightarrow \setR$ una función que determina la densidad en cada punto del espacio ocupado por cierto sólido.
La integral de $\rho (x,y,z)$ sobre el volumen $V$ calcula la masa del sólido.
\begin{equation*}
    m = \iiint_V \rho (x,y,z) \, \partial x \, \partial y \, \partial z
\end{equation*}


\section{Integrales sobre curvas y superficies}

Las integrales curvilíneas e integrales de superficie son integrales dobles o triples.
Pero como se evalúan sobre curvas o superficies, las variables con respecto a las que se integra se reducen a una o dos, respectivamente.

\begin{mdframed}[style=DefinitionFrame]
    \begin{defn}
    \end{defn}
    \cusTi{Diferencial de distancia}
    \begin{equation*}
        \dif \Vec{s} = \versor{t} \, \dif s = \left( \partial x, \partial y, \partial z \right)
    \end{equation*}
\end{mdframed}

\begin{mdframed}[style=DefinitionFrame]
    \begin{defn}
    \end{defn}
    \cusTi{Elemento de arco}
    \begin{equation*}
        \dif s = \nnorm{ \dfrac{\dif}{\dif t} \Vec{c}(t) } \dif t = \nnorm{ \dif \Vec{s} }
    \end{equation*}
\end{mdframed}

\begin{mdframed}[style=DefinitionFrame]
    \begin{defn}
    \end{defn}
    \cusTi{Diferencial de área}
    \begin{equation*}
        \dif \Vec{S} = \versor{n} \, \dif S = \dfrac{\grad S(x,y,z)}{\norm{ \dfrac{\partial}{\partial z} S(x,y,z) } } \, \partial x \, \partial y
    \end{equation*}
\end{mdframed}

\begin{mdframed}[style=DefinitionFrame]
    \begin{defn}
    \end{defn}
    \cusTi{Elemento de área}
    \begin{equation*}
        \dif S = \nnorm{ \dfrac{\partial}{\partial u} \Vec{s}(u,v) \times \dfrac{\partial}{\partial v} \Vec{s}(u,v) } \, \partial u \, \partial v = \nnorm{ \dif \Vec{S} }
    \end{equation*}
\end{mdframed}


\concept{Integrales de tipo 1}

Las integrales sobre curvas y superficies de tipo 1 se realizan sobre campos escalares.

\begin{mdframed}[style=DefinitionFrame]
    \begin{defn}
    \end{defn}
    \cusTi{Integral curvilínea de tipo 1}
    \begin{equation*}
        \int_C f \, \dif s = \int_{t_0}^{t_1} f \big( \Vec{c}(t) \big) \, \nnorm{ \dfrac{\dif}{\dif t} \Vec{c}(t) } \, \dif t
    \end{equation*}
\end{mdframed}

\begin{mdframed}[style=DefinitionFrame]
    \begin{defn}
    \end{defn}
    \cusTi{Integral de superficie de tipo 1}
    \begin{equation*}
        \int_S f \, \dif S = \iint f \big( \Vec{s}(u,v) \big) \, \nnorm{ \Vec{n} } \, \partial u \, \partial v
    \end{equation*}
\end{mdframed}


\concept{Integrales de tipo 2}

Las integrales sobre curvas y superficies de tipo 2 se realizan sobre campos vectoriales.

El trabajo es la cantidad de energía que se necesita para que una partícula recorra un arco de curva.
El movimiento puede darse en 2 o 3 dimensiones espaciales, pero en ambos casos se estaría calculando el trabajo sobre una curva.

\begin{mdframed}[style=DefinitionFrame]
    \begin{defn}
        \label{defn:type2Int}
    \end{defn}
    \cusTi{Integral curvilínea de tipo 2 (Trabajo)}
    \begin{equation*}
        W = \int_C \Vec{F} \cdot \dif \Vec{s}
        = \int_{t_0}^{t_1} \Vec{F} \big( \Vec{c}(t) \big) \, \dfrac{\dif}{\dif t} \Vec{c}(t) \, \dif t
    \end{equation*}
\end{mdframed}

El flujo en 2 dimensiones espaciales mide que tanto ``Caudal de Fuerza'' fluye a través de una curva en cierto intervalo de tiempo.

\begin{mdframed}[style=DefinitionFrame]
    \begin{defn}
    \end{defn}
    \cusTi{Integral curvilínea de tipo 2 (Flujo 2D)}
    \begin{equation*}
        \Phi = \int_C \Vec{F} \cdot \dif \Vec{S}
        = \int_{t_0}^{t_1} \Vec{F} \big( \Vec{c}(t) \big) \cdot \versor{n} \, \dif s
    \end{equation*}
\end{mdframed}

El flujo en 3 dimensiones espaciales mide que tanto ``Caudal de Fuerza'' fluye a través de una superficie en cierto intervalo de tiempo.

\begin{mdframed}[style=DefinitionFrame]
    \begin{defn}
    \end{defn}
    \cusTi{Integral de superficie de tipo 2 (Flujo 3D)}
    \begin{equation*}
        \Phi = \int_S \Vec{F} \cdot \dif \Vec{S}
        = \iint \Vec{F} \big( \Vec{s}(u,v) \big) \cdot \versor{n} \, \dif S
    \end{equation*}
\end{mdframed}