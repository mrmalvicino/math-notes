\chapter{Álgebra elemental}

\begin{mdframed}[style=DefinitionFrame]
    \begin{defn}
    \end{defn}
    \cusTi{Matriz diagonal}
    \cusTe{Una matriz se dice diagonal si $a_{ij}=0$ cuando $i \neq j$}
    \begin{equation*}
        A =
        \begin{pmatrix}
            a_{11} & 0 & \dots & 0
            \\
            0 & a_{22} & \dots & 0
            \\
            \vdots & \vdots & \ddots & \vdots
            \\
            0 & 0 & \dots & a_{\mth\nth}
        \end{pmatrix}
    \end{equation*}
\end{mdframed}

\begin{mdframed}[style=PropertyFrame]
    \begin{prop}
    \end{prop}
    El determinante de una matriz triangulada inferior o superiormente es el producto de su diagonal principal.
\end{mdframed}

\begin{mdframed}[style=PropertyFrame]
    \begin{prop}
    \end{prop}
    Si el determinante de una matriz equivalente es distinto de cero, entonces el determinante de la matriz original también lo es.
\end{mdframed}

\begin{mdframed}[style=DefinitionFrame]
    \begin{defn}
    \end{defn}
    \cusTi{Rango de una matriz}
    \cusTe{Se denota como $\operatorname{Rg}(M)$ la cantidad de filas o columnas LI de una matriz una vez triangulada.}
\end{mdframed}

\begin{center}
    \def\svgwidth{\linewidth}
    \input{./images/alg-sistemas.pdf_tex}
\end{center}


\section{Teo. de Rouché–Frobenius}

Dado un sistema lineal de $\mth$ ecuaciones con $\nth$ incógnitas:
\begin{equation*}
    \left\{
    \begin{matrix}
        a_{1,1} \, x_1 + a_{1,2} \, x_2 + \dots + a_{1,\nth} \, x_\nth & = & b_1
        \\
        a_{2,1} \, x_1 + a_{2,2} \, x_2 + \dots + a_{2,\nth} \, x_\nth & = & b_2
        \\
        \vdots & \vdots & \vdots
        \\
        a_{\mth,1} \, x_1 + a_{\mth,2} \, x_2 + \dots + a_{\mth,\nth} \, x_\nth & = & b_\mth
    \end{matrix}
    \right.
\end{equation*}

Representado de forma matricial por:
\begin{gather*}
    A \cdot \Vec{x} = \Vec{b}
    \\
    \begin{pmatrix}
        a_{1,1} & \dots & a_{1,\nth}
        \\
        \vdots & \ddots & \vdots
        \\
        a_{\mth,1} & \dots & a_{\mth,\nth}
    \end{pmatrix}
    \cdot
    \begin{pmatrix}
        x_1
        \\
        \vdots
        \\
        x_\nth
    \end{pmatrix}
    =
    \begin{pmatrix}
        b_1
        \\
        \vdots
        \\
        b_\mth
    \end{pmatrix}
\end{gather*}

Entonces el sistema lineal es compatible si el rango de la matriz de coeficientes es igual al rango de la matriz ampliada:
\begin{equation*}
    \operatorname{rg}(A) = \operatorname{rg}\left(A\Big|\Vec{b}\right)
\end{equation*}

Y el conjunto de soluciones $S$ se puede expresar de forma paramétrica por $\nth-\operatorname{rg}(A)$ parámetros.
Particularmente, si $\operatorname{rg}(A)=\nth$ entonces la solución es un único punto.

Más formalmente, se dice que el conjunto de soluciones forman un subespacio afín $S\in\setR^\nth$ que verifica:
\begin{equation*}
    \operatorname{dim}(S) = \nth-\operatorname{rg}(A)
\end{equation*}